
% Default to the notebook output style

    


% Inherit from the specified cell style.




    
\documentclass[11pt]{article}

    
    
    \usepackage[T1]{fontenc}
    % Nicer default font (+ math font) than Computer Modern for most use cases
    \usepackage{mathpazo}

    % Basic figure setup, for now with no caption control since it's done
    % automatically by Pandoc (which extracts ![](path) syntax from Markdown).
    \usepackage{graphicx}
    % We will generate all images so they have a width \maxwidth. This means
    % that they will get their normal width if they fit onto the page, but
    % are scaled down if they would overflow the margins.
    \makeatletter
    \def\maxwidth{\ifdim\Gin@nat@width>\linewidth\linewidth
    \else\Gin@nat@width\fi}
    \makeatother
    \let\Oldincludegraphics\includegraphics
    % Set max figure width to be 80% of text width, for now hardcoded.
    \renewcommand{\includegraphics}[1]{\Oldincludegraphics[width=.8\maxwidth]{#1}}
    % Ensure that by default, figures have no caption (until we provide a
    % proper Figure object with a Caption API and a way to capture that
    % in the conversion process - todo).
    \usepackage{caption}
    \DeclareCaptionLabelFormat{nolabel}{}
    \captionsetup{labelformat=nolabel}

    \usepackage{adjustbox} % Used to constrain images to a maximum size 
    \usepackage{xcolor} % Allow colors to be defined
    \usepackage{enumerate} % Needed for markdown enumerations to work
    \usepackage{geometry} % Used to adjust the document margins
    \usepackage{amsmath} % Equations
    \usepackage{amssymb} % Equations
    \usepackage{textcomp} % defines textquotesingle
    % Hack from http://tex.stackexchange.com/a/47451/13684:
    \AtBeginDocument{%
        \def\PYZsq{\textquotesingle}% Upright quotes in Pygmentized code
    }
    \usepackage{upquote} % Upright quotes for verbatim code
    \usepackage{eurosym} % defines \euro
    \usepackage[mathletters]{ucs} % Extended unicode (utf-8) support
    \usepackage[utf8x]{inputenc} % Allow utf-8 characters in the tex document
    \usepackage{fancyvrb} % verbatim replacement that allows latex
    \usepackage{grffile} % extends the file name processing of package graphics 
                         % to support a larger range 
    % The hyperref package gives us a pdf with properly built
    % internal navigation ('pdf bookmarks' for the table of contents,
    % internal cross-reference links, web links for URLs, etc.)
    \usepackage{hyperref}
    \usepackage{longtable} % longtable support required by pandoc >1.10
    \usepackage{booktabs}  % table support for pandoc > 1.12.2
    \usepackage[inline]{enumitem} % IRkernel/repr support (it uses the enumerate* environment)
    \usepackage[normalem]{ulem} % ulem is needed to support strikethroughs (\sout)
                                % normalem makes italics be italics, not underlines
    

    
    
    % Colors for the hyperref package
    \definecolor{urlcolor}{rgb}{0,.145,.698}
    \definecolor{linkcolor}{rgb}{.71,0.21,0.01}
    \definecolor{citecolor}{rgb}{.12,.54,.11}

    % ANSI colors
    \definecolor{ansi-black}{HTML}{3E424D}
    \definecolor{ansi-black-intense}{HTML}{282C36}
    \definecolor{ansi-red}{HTML}{E75C58}
    \definecolor{ansi-red-intense}{HTML}{B22B31}
    \definecolor{ansi-green}{HTML}{00A250}
    \definecolor{ansi-green-intense}{HTML}{007427}
    \definecolor{ansi-yellow}{HTML}{DDB62B}
    \definecolor{ansi-yellow-intense}{HTML}{B27D12}
    \definecolor{ansi-blue}{HTML}{208FFB}
    \definecolor{ansi-blue-intense}{HTML}{0065CA}
    \definecolor{ansi-magenta}{HTML}{D160C4}
    \definecolor{ansi-magenta-intense}{HTML}{A03196}
    \definecolor{ansi-cyan}{HTML}{60C6C8}
    \definecolor{ansi-cyan-intense}{HTML}{258F8F}
    \definecolor{ansi-white}{HTML}{C5C1B4}
    \definecolor{ansi-white-intense}{HTML}{A1A6B2}

    % commands and environments needed by pandoc snippets
    % extracted from the output of `pandoc -s`
    \providecommand{\tightlist}{%
      \setlength{\itemsep}{0pt}\setlength{\parskip}{0pt}}
    \DefineVerbatimEnvironment{Highlighting}{Verbatim}{commandchars=\\\{\}}
    % Add ',fontsize=\small' for more characters per line
    \newenvironment{Shaded}{}{}
    \newcommand{\KeywordTok}[1]{\textcolor[rgb]{0.00,0.44,0.13}{\textbf{{#1}}}}
    \newcommand{\DataTypeTok}[1]{\textcolor[rgb]{0.56,0.13,0.00}{{#1}}}
    \newcommand{\DecValTok}[1]{\textcolor[rgb]{0.25,0.63,0.44}{{#1}}}
    \newcommand{\BaseNTok}[1]{\textcolor[rgb]{0.25,0.63,0.44}{{#1}}}
    \newcommand{\FloatTok}[1]{\textcolor[rgb]{0.25,0.63,0.44}{{#1}}}
    \newcommand{\CharTok}[1]{\textcolor[rgb]{0.25,0.44,0.63}{{#1}}}
    \newcommand{\StringTok}[1]{\textcolor[rgb]{0.25,0.44,0.63}{{#1}}}
    \newcommand{\CommentTok}[1]{\textcolor[rgb]{0.38,0.63,0.69}{\textit{{#1}}}}
    \newcommand{\OtherTok}[1]{\textcolor[rgb]{0.00,0.44,0.13}{{#1}}}
    \newcommand{\AlertTok}[1]{\textcolor[rgb]{1.00,0.00,0.00}{\textbf{{#1}}}}
    \newcommand{\FunctionTok}[1]{\textcolor[rgb]{0.02,0.16,0.49}{{#1}}}
    \newcommand{\RegionMarkerTok}[1]{{#1}}
    \newcommand{\ErrorTok}[1]{\textcolor[rgb]{1.00,0.00,0.00}{\textbf{{#1}}}}
    \newcommand{\NormalTok}[1]{{#1}}
    
    % Additional commands for more recent versions of Pandoc
    \newcommand{\ConstantTok}[1]{\textcolor[rgb]{0.53,0.00,0.00}{{#1}}}
    \newcommand{\SpecialCharTok}[1]{\textcolor[rgb]{0.25,0.44,0.63}{{#1}}}
    \newcommand{\VerbatimStringTok}[1]{\textcolor[rgb]{0.25,0.44,0.63}{{#1}}}
    \newcommand{\SpecialStringTok}[1]{\textcolor[rgb]{0.73,0.40,0.53}{{#1}}}
    \newcommand{\ImportTok}[1]{{#1}}
    \newcommand{\DocumentationTok}[1]{\textcolor[rgb]{0.73,0.13,0.13}{\textit{{#1}}}}
    \newcommand{\AnnotationTok}[1]{\textcolor[rgb]{0.38,0.63,0.69}{\textbf{\textit{{#1}}}}}
    \newcommand{\CommentVarTok}[1]{\textcolor[rgb]{0.38,0.63,0.69}{\textbf{\textit{{#1}}}}}
    \newcommand{\VariableTok}[1]{\textcolor[rgb]{0.10,0.09,0.49}{{#1}}}
    \newcommand{\ControlFlowTok}[1]{\textcolor[rgb]{0.00,0.44,0.13}{\textbf{{#1}}}}
    \newcommand{\OperatorTok}[1]{\textcolor[rgb]{0.40,0.40,0.40}{{#1}}}
    \newcommand{\BuiltInTok}[1]{{#1}}
    \newcommand{\ExtensionTok}[1]{{#1}}
    \newcommand{\PreprocessorTok}[1]{\textcolor[rgb]{0.74,0.48,0.00}{{#1}}}
    \newcommand{\AttributeTok}[1]{\textcolor[rgb]{0.49,0.56,0.16}{{#1}}}
    \newcommand{\InformationTok}[1]{\textcolor[rgb]{0.38,0.63,0.69}{\textbf{\textit{{#1}}}}}
    \newcommand{\WarningTok}[1]{\textcolor[rgb]{0.38,0.63,0.69}{\textbf{\textit{{#1}}}}}
    
    
    % Define a nice break command that doesn't care if a line doesn't already
    % exist.
    \def\br{\hspace*{\fill} \\* }
    % Math Jax compatability definitions
    \def\gt{>}
    \def\lt{<}
    % Document parameters
    \title{spacepython}
    
    
    

    % Pygments definitions
    
\makeatletter
\def\PY@reset{\let\PY@it=\relax \let\PY@bf=\relax%
    \let\PY@ul=\relax \let\PY@tc=\relax%
    \let\PY@bc=\relax \let\PY@ff=\relax}
\def\PY@tok#1{\csname PY@tok@#1\endcsname}
\def\PY@toks#1+{\ifx\relax#1\empty\else%
    \PY@tok{#1}\expandafter\PY@toks\fi}
\def\PY@do#1{\PY@bc{\PY@tc{\PY@ul{%
    \PY@it{\PY@bf{\PY@ff{#1}}}}}}}
\def\PY#1#2{\PY@reset\PY@toks#1+\relax+\PY@do{#2}}

\expandafter\def\csname PY@tok@w\endcsname{\def\PY@tc##1{\textcolor[rgb]{0.73,0.73,0.73}{##1}}}
\expandafter\def\csname PY@tok@c\endcsname{\let\PY@it=\textit\def\PY@tc##1{\textcolor[rgb]{0.25,0.50,0.50}{##1}}}
\expandafter\def\csname PY@tok@cp\endcsname{\def\PY@tc##1{\textcolor[rgb]{0.74,0.48,0.00}{##1}}}
\expandafter\def\csname PY@tok@k\endcsname{\let\PY@bf=\textbf\def\PY@tc##1{\textcolor[rgb]{0.00,0.50,0.00}{##1}}}
\expandafter\def\csname PY@tok@kp\endcsname{\def\PY@tc##1{\textcolor[rgb]{0.00,0.50,0.00}{##1}}}
\expandafter\def\csname PY@tok@kt\endcsname{\def\PY@tc##1{\textcolor[rgb]{0.69,0.00,0.25}{##1}}}
\expandafter\def\csname PY@tok@o\endcsname{\def\PY@tc##1{\textcolor[rgb]{0.40,0.40,0.40}{##1}}}
\expandafter\def\csname PY@tok@ow\endcsname{\let\PY@bf=\textbf\def\PY@tc##1{\textcolor[rgb]{0.67,0.13,1.00}{##1}}}
\expandafter\def\csname PY@tok@nb\endcsname{\def\PY@tc##1{\textcolor[rgb]{0.00,0.50,0.00}{##1}}}
\expandafter\def\csname PY@tok@nf\endcsname{\def\PY@tc##1{\textcolor[rgb]{0.00,0.00,1.00}{##1}}}
\expandafter\def\csname PY@tok@nc\endcsname{\let\PY@bf=\textbf\def\PY@tc##1{\textcolor[rgb]{0.00,0.00,1.00}{##1}}}
\expandafter\def\csname PY@tok@nn\endcsname{\let\PY@bf=\textbf\def\PY@tc##1{\textcolor[rgb]{0.00,0.00,1.00}{##1}}}
\expandafter\def\csname PY@tok@ne\endcsname{\let\PY@bf=\textbf\def\PY@tc##1{\textcolor[rgb]{0.82,0.25,0.23}{##1}}}
\expandafter\def\csname PY@tok@nv\endcsname{\def\PY@tc##1{\textcolor[rgb]{0.10,0.09,0.49}{##1}}}
\expandafter\def\csname PY@tok@no\endcsname{\def\PY@tc##1{\textcolor[rgb]{0.53,0.00,0.00}{##1}}}
\expandafter\def\csname PY@tok@nl\endcsname{\def\PY@tc##1{\textcolor[rgb]{0.63,0.63,0.00}{##1}}}
\expandafter\def\csname PY@tok@ni\endcsname{\let\PY@bf=\textbf\def\PY@tc##1{\textcolor[rgb]{0.60,0.60,0.60}{##1}}}
\expandafter\def\csname PY@tok@na\endcsname{\def\PY@tc##1{\textcolor[rgb]{0.49,0.56,0.16}{##1}}}
\expandafter\def\csname PY@tok@nt\endcsname{\let\PY@bf=\textbf\def\PY@tc##1{\textcolor[rgb]{0.00,0.50,0.00}{##1}}}
\expandafter\def\csname PY@tok@nd\endcsname{\def\PY@tc##1{\textcolor[rgb]{0.67,0.13,1.00}{##1}}}
\expandafter\def\csname PY@tok@s\endcsname{\def\PY@tc##1{\textcolor[rgb]{0.73,0.13,0.13}{##1}}}
\expandafter\def\csname PY@tok@sd\endcsname{\let\PY@it=\textit\def\PY@tc##1{\textcolor[rgb]{0.73,0.13,0.13}{##1}}}
\expandafter\def\csname PY@tok@si\endcsname{\let\PY@bf=\textbf\def\PY@tc##1{\textcolor[rgb]{0.73,0.40,0.53}{##1}}}
\expandafter\def\csname PY@tok@se\endcsname{\let\PY@bf=\textbf\def\PY@tc##1{\textcolor[rgb]{0.73,0.40,0.13}{##1}}}
\expandafter\def\csname PY@tok@sr\endcsname{\def\PY@tc##1{\textcolor[rgb]{0.73,0.40,0.53}{##1}}}
\expandafter\def\csname PY@tok@ss\endcsname{\def\PY@tc##1{\textcolor[rgb]{0.10,0.09,0.49}{##1}}}
\expandafter\def\csname PY@tok@sx\endcsname{\def\PY@tc##1{\textcolor[rgb]{0.00,0.50,0.00}{##1}}}
\expandafter\def\csname PY@tok@m\endcsname{\def\PY@tc##1{\textcolor[rgb]{0.40,0.40,0.40}{##1}}}
\expandafter\def\csname PY@tok@gh\endcsname{\let\PY@bf=\textbf\def\PY@tc##1{\textcolor[rgb]{0.00,0.00,0.50}{##1}}}
\expandafter\def\csname PY@tok@gu\endcsname{\let\PY@bf=\textbf\def\PY@tc##1{\textcolor[rgb]{0.50,0.00,0.50}{##1}}}
\expandafter\def\csname PY@tok@gd\endcsname{\def\PY@tc##1{\textcolor[rgb]{0.63,0.00,0.00}{##1}}}
\expandafter\def\csname PY@tok@gi\endcsname{\def\PY@tc##1{\textcolor[rgb]{0.00,0.63,0.00}{##1}}}
\expandafter\def\csname PY@tok@gr\endcsname{\def\PY@tc##1{\textcolor[rgb]{1.00,0.00,0.00}{##1}}}
\expandafter\def\csname PY@tok@ge\endcsname{\let\PY@it=\textit}
\expandafter\def\csname PY@tok@gs\endcsname{\let\PY@bf=\textbf}
\expandafter\def\csname PY@tok@gp\endcsname{\let\PY@bf=\textbf\def\PY@tc##1{\textcolor[rgb]{0.00,0.00,0.50}{##1}}}
\expandafter\def\csname PY@tok@go\endcsname{\def\PY@tc##1{\textcolor[rgb]{0.53,0.53,0.53}{##1}}}
\expandafter\def\csname PY@tok@gt\endcsname{\def\PY@tc##1{\textcolor[rgb]{0.00,0.27,0.87}{##1}}}
\expandafter\def\csname PY@tok@err\endcsname{\def\PY@bc##1{\setlength{\fboxsep}{0pt}\fcolorbox[rgb]{1.00,0.00,0.00}{1,1,1}{\strut ##1}}}
\expandafter\def\csname PY@tok@kc\endcsname{\let\PY@bf=\textbf\def\PY@tc##1{\textcolor[rgb]{0.00,0.50,0.00}{##1}}}
\expandafter\def\csname PY@tok@kd\endcsname{\let\PY@bf=\textbf\def\PY@tc##1{\textcolor[rgb]{0.00,0.50,0.00}{##1}}}
\expandafter\def\csname PY@tok@kn\endcsname{\let\PY@bf=\textbf\def\PY@tc##1{\textcolor[rgb]{0.00,0.50,0.00}{##1}}}
\expandafter\def\csname PY@tok@kr\endcsname{\let\PY@bf=\textbf\def\PY@tc##1{\textcolor[rgb]{0.00,0.50,0.00}{##1}}}
\expandafter\def\csname PY@tok@bp\endcsname{\def\PY@tc##1{\textcolor[rgb]{0.00,0.50,0.00}{##1}}}
\expandafter\def\csname PY@tok@fm\endcsname{\def\PY@tc##1{\textcolor[rgb]{0.00,0.00,1.00}{##1}}}
\expandafter\def\csname PY@tok@vc\endcsname{\def\PY@tc##1{\textcolor[rgb]{0.10,0.09,0.49}{##1}}}
\expandafter\def\csname PY@tok@vg\endcsname{\def\PY@tc##1{\textcolor[rgb]{0.10,0.09,0.49}{##1}}}
\expandafter\def\csname PY@tok@vi\endcsname{\def\PY@tc##1{\textcolor[rgb]{0.10,0.09,0.49}{##1}}}
\expandafter\def\csname PY@tok@vm\endcsname{\def\PY@tc##1{\textcolor[rgb]{0.10,0.09,0.49}{##1}}}
\expandafter\def\csname PY@tok@sa\endcsname{\def\PY@tc##1{\textcolor[rgb]{0.73,0.13,0.13}{##1}}}
\expandafter\def\csname PY@tok@sb\endcsname{\def\PY@tc##1{\textcolor[rgb]{0.73,0.13,0.13}{##1}}}
\expandafter\def\csname PY@tok@sc\endcsname{\def\PY@tc##1{\textcolor[rgb]{0.73,0.13,0.13}{##1}}}
\expandafter\def\csname PY@tok@dl\endcsname{\def\PY@tc##1{\textcolor[rgb]{0.73,0.13,0.13}{##1}}}
\expandafter\def\csname PY@tok@s2\endcsname{\def\PY@tc##1{\textcolor[rgb]{0.73,0.13,0.13}{##1}}}
\expandafter\def\csname PY@tok@sh\endcsname{\def\PY@tc##1{\textcolor[rgb]{0.73,0.13,0.13}{##1}}}
\expandafter\def\csname PY@tok@s1\endcsname{\def\PY@tc##1{\textcolor[rgb]{0.73,0.13,0.13}{##1}}}
\expandafter\def\csname PY@tok@mb\endcsname{\def\PY@tc##1{\textcolor[rgb]{0.40,0.40,0.40}{##1}}}
\expandafter\def\csname PY@tok@mf\endcsname{\def\PY@tc##1{\textcolor[rgb]{0.40,0.40,0.40}{##1}}}
\expandafter\def\csname PY@tok@mh\endcsname{\def\PY@tc##1{\textcolor[rgb]{0.40,0.40,0.40}{##1}}}
\expandafter\def\csname PY@tok@mi\endcsname{\def\PY@tc##1{\textcolor[rgb]{0.40,0.40,0.40}{##1}}}
\expandafter\def\csname PY@tok@il\endcsname{\def\PY@tc##1{\textcolor[rgb]{0.40,0.40,0.40}{##1}}}
\expandafter\def\csname PY@tok@mo\endcsname{\def\PY@tc##1{\textcolor[rgb]{0.40,0.40,0.40}{##1}}}
\expandafter\def\csname PY@tok@ch\endcsname{\let\PY@it=\textit\def\PY@tc##1{\textcolor[rgb]{0.25,0.50,0.50}{##1}}}
\expandafter\def\csname PY@tok@cm\endcsname{\let\PY@it=\textit\def\PY@tc##1{\textcolor[rgb]{0.25,0.50,0.50}{##1}}}
\expandafter\def\csname PY@tok@cpf\endcsname{\let\PY@it=\textit\def\PY@tc##1{\textcolor[rgb]{0.25,0.50,0.50}{##1}}}
\expandafter\def\csname PY@tok@c1\endcsname{\let\PY@it=\textit\def\PY@tc##1{\textcolor[rgb]{0.25,0.50,0.50}{##1}}}
\expandafter\def\csname PY@tok@cs\endcsname{\let\PY@it=\textit\def\PY@tc##1{\textcolor[rgb]{0.25,0.50,0.50}{##1}}}

\def\PYZbs{\char`\\}
\def\PYZus{\char`\_}
\def\PYZob{\char`\{}
\def\PYZcb{\char`\}}
\def\PYZca{\char`\^}
\def\PYZam{\char`\&}
\def\PYZlt{\char`\<}
\def\PYZgt{\char`\>}
\def\PYZsh{\char`\#}
\def\PYZpc{\char`\%}
\def\PYZdl{\char`\$}
\def\PYZhy{\char`\-}
\def\PYZsq{\char`\'}
\def\PYZdq{\char`\"}
\def\PYZti{\char`\~}
% for compatibility with earlier versions
\def\PYZat{@}
\def\PYZlb{[}
\def\PYZrb{]}
\makeatother


    % Exact colors from NB
    \definecolor{incolor}{rgb}{0.0, 0.0, 0.5}
    \definecolor{outcolor}{rgb}{0.545, 0.0, 0.0}



    
    % Prevent overflowing lines due to hard-to-break entities
    \sloppy 
    % Setup hyperref package
    \hypersetup{
      breaklinks=true,  % so long urls are correctly broken across lines
      colorlinks=true,
      urlcolor=urlcolor,
      linkcolor=linkcolor,
      citecolor=citecolor,
      }
    % Slightly bigger margins than the latex defaults
    
    \geometry{verbose,tmargin=1in,bmargin=1in,lmargin=1in,rmargin=1in}
    
    

    \begin{document}
    
    
    \maketitle
    
    

    
    \begin{Verbatim}[commandchars=\\\{\}]
{\color{incolor}In [{\color{incolor}1}]:} \PY{k+kn}{import} \PY{n+nn}{warnings}
        \PY{n}{warnings}\PY{o}{.}\PY{n}{filterwarnings}\PY{p}{(}\PY{l+s+s2}{\PYZdq{}}\PY{l+s+s2}{ignore}\PY{l+s+s2}{\PYZdq{}}\PY{p}{)}
        \PY{o}{\PYZpc{}}\PY{k}{matplotlib} inline
\end{Verbatim}


    \hypertarget{python-for-space-dummies}{%
\section{Python for space dummies}\label{python-for-space-dummies}}

\begin{itemize}
\tightlist
\item
  Basics: Lists, for loops, importing modules, getting help
\item
  Two standard library examples: datetime (handling time), pickle (save
  to disk)
\item
  Numpy (array data type)
\item
  Sympy (symbolic calculation)
\item
  Matplotlib (plotting)
\item
  Scipy (ecosystem of math/science/engineering tools)
\item
  Pandas (data handling)
\item
  Other useful libraries
\item
  Working with CDF files
\item
  Space / upper atmosphere models: MSIS, HWM, IRI, IGRF, CHAOS,
  Tsyganenko, AMPS
\item
  Magnetic coordinates
\item
  Pytt/Pyrkeland
\end{itemize}

    \hypertarget{basics}{%
\subsection{Basics }\label{basics}}

Get Python + most relevant modules + package manager from
https://www.anaconda.com

This document is written with JuPyter notebook.

Usually I write code in Sublime Text (a text editor,
https://www.sublimetext.com/3), and run the code using an \emph{ipython}
terminal

    \hypertarget{python-2-or-3}{%
\subsubsection{Python 2 or 3?}\label{python-2-or-3}}

Python 3.

There are still some (very few) modules that only work in Python 2, but
I expect that gradually this will reverse.

The difference between Python 2 and 3 is very small

    \hypertarget{lists}{%
\subsubsection{Lists}\label{lists}}

To get started, I will show some examples with a built-in container
type, \emph{list}:

    \begin{Verbatim}[commandchars=\\\{\}]
{\color{incolor}In [{\color{incolor}2}]:} \PY{n}{my\PYZus{}list} \PY{o}{=} \PY{p}{[}\PY{l+m+mi}{2}\PY{p}{,} \PY{l+m+mi}{3}\PY{p}{,} \PY{l+m+mi}{7}\PY{p}{,} \PY{l+m+mi}{11}\PY{p}{]}
        \PY{n+nb}{print}\PY{p}{(}\PY{n}{my\PYZus{}list}\PY{p}{)}
\end{Verbatim}


    \begin{Verbatim}[commandchars=\\\{\}]
[2, 3, 7, 11]

    \end{Verbatim}

    \begin{Verbatim}[commandchars=\\\{\}]
{\color{incolor}In [{\color{incolor}3}]:} \PY{c+c1}{\PYZsh{} They look like vectors/arrays but this is not the case:}
        \PY{n+nb}{print}\PY{p}{(}\PY{n}{my\PYZus{}list} \PY{o}{*} \PY{l+m+mi}{2}\PY{p}{)}
\end{Verbatim}


    \begin{Verbatim}[commandchars=\\\{\}]
[2, 3, 7, 11, 2, 3, 7, 11]

    \end{Verbatim}

    \begin{Verbatim}[commandchars=\\\{\}]
{\color{incolor}In [{\color{incolor}4}]:} \PY{n+nb}{print}\PY{p}{(}\PY{n}{my\PYZus{}list} \PY{o}{+} \PY{p}{[}\PY{l+m+mi}{100}\PY{p}{,} \PY{l+m+mi}{101}\PY{p}{]}\PY{p}{)}
\end{Verbatim}


    \begin{Verbatim}[commandchars=\\\{\}]
[2, 3, 7, 11, 100, 101]

    \end{Verbatim}

    Their elements can be anything:

    \begin{Verbatim}[commandchars=\\\{\}]
{\color{incolor}In [{\color{incolor}5}]:} \PY{n}{my\PYZus{}new\PYZus{}list} \PY{o}{=} \PY{p}{[}\PY{l+m+mi}{1}\PY{p}{,} \PY{l+s+s1}{\PYZsq{}}\PY{l+s+s1}{a string}\PY{l+s+s1}{\PYZsq{}}\PY{p}{,} \PY{l+s+s2}{\PYZdq{}}\PY{l+s+s2}{another string}\PY{l+s+s2}{\PYZdq{}}\PY{p}{,} \PY{n}{my\PYZus{}list}\PY{p}{,} \PY{n+nb}{print}\PY{p}{]}
        \PY{n+nb}{print}\PY{p}{(}\PY{n}{my\PYZus{}new\PYZus{}list}\PY{p}{)}
\end{Verbatim}


    \begin{Verbatim}[commandchars=\\\{\}]
[1, 'a string', 'another string', [2, 3, 7, 11], <built-in function print>]

    \end{Verbatim}

    Nice, but pretty useless if we want normal vector/matrix operations. We
will fix that with the Numpy module

    \hypertarget{for-loops}{%
\subsubsection{For loops }\label{for-loops}}

Example of for loop syntax in Python, in order to illustrate the
structuring by indentation

    \begin{Verbatim}[commandchars=\\\{\}]
{\color{incolor}In [{\color{incolor}6}]:} \PY{k}{for} \PY{n}{item} \PY{o+ow}{in} \PY{n}{my\PYZus{}list}\PY{p}{:}
            \PY{n}{doubled} \PY{o}{=} \PY{n}{item} \PY{o}{*} \PY{l+m+mi}{2}
            \PY{n+nb}{print}\PY{p}{(} \PY{n}{doubled} \PY{p}{)}
\end{Verbatim}


    \begin{Verbatim}[commandchars=\\\{\}]
4
6
14
22

    \end{Verbatim}

    \hypertarget{import-modules}{%
\subsubsection{Import modules}\label{import-modules}}

    \begin{Verbatim}[commandchars=\\\{\}]
{\color{incolor}In [{\color{incolor}7}]:} \PY{k+kn}{import} \PY{n+nn}{numpy}
        \PY{n+nb}{print}\PY{p}{(}\PY{n}{numpy}\PY{o}{.}\PY{n}{sqrt}\PY{p}{(}\PY{l+m+mi}{2}\PY{p}{)}\PY{p}{)}
\end{Verbatim}


    \begin{Verbatim}[commandchars=\\\{\}]
1.4142135623730951

    \end{Verbatim}

    \begin{Verbatim}[commandchars=\\\{\}]
{\color{incolor}In [{\color{incolor}8}]:} \PY{k+kn}{from} \PY{n+nn}{numpy} \PY{k}{import} \PY{n}{sqrt}
        \PY{n+nb}{print}\PY{p}{(}\PY{n}{sqrt}\PY{p}{(}\PY{l+m+mi}{2}\PY{p}{)}\PY{p}{)}
\end{Verbatim}


    \begin{Verbatim}[commandchars=\\\{\}]
1.4142135623730951

    \end{Verbatim}

    \begin{Verbatim}[commandchars=\\\{\}]
{\color{incolor}In [{\color{incolor}9}]:} \PY{k+kn}{import} \PY{n+nn}{numpy} \PY{k}{as} \PY{n+nn}{np}
        \PY{n+nb}{print}\PY{p}{(}\PY{n}{np}\PY{o}{.}\PY{n}{sqrt}\PY{p}{(}\PY{l+m+mi}{2}\PY{p}{)}\PY{p}{)}
\end{Verbatim}


    \begin{Verbatim}[commandchars=\\\{\}]
1.4142135623730951

    \end{Verbatim}

    \begin{Verbatim}[commandchars=\\\{\}]
{\color{incolor}In [{\color{incolor}10}]:} \PY{k+kn}{from} \PY{n+nn}{numpy} \PY{k}{import} \PY{o}{*}
         \PY{n+nb}{print}\PY{p}{(}\PY{n}{rad2deg}\PY{p}{(}\PY{n}{pi}\PY{p}{)}\PY{p}{)}
\end{Verbatim}


    \begin{Verbatim}[commandchars=\\\{\}]
180.0

    \end{Verbatim}

    \hypertarget{how-to-get-help}{%
\subsubsection{How to get help}\label{how-to-get-help}}

    \begin{Verbatim}[commandchars=\\\{\}]
{\color{incolor}In [{\color{incolor}11}]:} \PY{k}{def} \PY{n+nf}{my\PYZus{}function}\PY{p}{(}\PY{n}{x}\PY{p}{)}\PY{p}{:}
             \PY{l+s+sd}{\PYZdq{}\PYZdq{}\PYZdq{} This is the doc\PYZhy{}string for my\PYZus{}function}
         \PY{l+s+sd}{    }
         \PY{l+s+sd}{    It returns pi }
         \PY{l+s+sd}{    ¯\PYZbs{}\PYZus{}(ツ)\PYZus{}/¯}
         \PY{l+s+sd}{    }
         \PY{l+s+sd}{    \PYZdq{}\PYZdq{}\PYZdq{}}
             \PY{k}{return} \PY{n}{np}\PY{o}{.}\PY{n}{pi}
\end{Verbatim}


    \begin{Verbatim}[commandchars=\\\{\}]
{\color{incolor}In [{\color{incolor}12}]:} \PY{n}{help}\PY{p}{(}\PY{n}{my\PYZus{}function}\PY{p}{)}
\end{Verbatim}


    \begin{Verbatim}[commandchars=\\\{\}]
Help on function my\_function in module \_\_main\_\_:

my\_function(x)
    This is the doc-string for my\_function
    
    It returns pi 
    ¯\textbackslash{}\_(ツ)\_/¯


    \end{Verbatim}

    \begin{Verbatim}[commandchars=\\\{\}]
{\color{incolor}In [{\color{incolor}13}]:} \PY{c+c1}{\PYZsh{} in ipython, this command prints the docstring:}
         my\PYZus{}function\PY{o}{?}
         
         \PY{c+c1}{\PYZsh{} and this command prints the source code:}
         my\PYZus{}function\PY{o}{??}
\end{Verbatim}


    \hypertarget{two-standard-library-modules}{%
\subsubsection{Two standard library
modules}\label{two-standard-library-modules}}

\begin{itemize}
\tightlist
\item
  datetime - for time handling 
\item
  pickle - for saving to disk
\end{itemize}

    \hypertarget{datetime}{%
\subsubsection{Datetime}\label{datetime}}

Standard library module for handling time

    \begin{Verbatim}[commandchars=\\\{\}]
{\color{incolor}In [{\color{incolor}14}]:} \PY{k+kn}{from} \PY{n+nn}{datetime} \PY{k}{import} \PY{n}{datetime}
         
         \PY{n}{bcss\PYZus{}kickoff} \PY{o}{=} \PY{n}{datetime}\PY{p}{(}\PY{l+m+mi}{2013}\PY{p}{,} \PY{l+m+mi}{3}\PY{p}{,} \PY{l+m+mi}{7}\PY{p}{,} \PY{l+m+mi}{9}\PY{p}{)}
         \PY{n}{time\PYZus{}since\PYZus{}kickoff} \PY{o}{=} \PY{n}{datetime}\PY{o}{.}\PY{n}{now}\PY{p}{(}\PY{p}{)} \PY{o}{\PYZhy{}} \PY{n}{bcss\PYZus{}kickoff}
         
         \PY{n+nb}{print}\PY{p}{(}\PY{l+s+s1}{\PYZsq{}}\PY{l+s+s1}{It is }\PY{l+s+si}{\PYZpc{}s}\PY{l+s+s1}{ seconds since BCSS kickoff}\PY{l+s+s1}{\PYZsq{}} \PY{o}{\PYZpc{}} 
               \PY{n+nb}{int}\PY{p}{(}\PY{n}{time\PYZus{}since\PYZus{}kickoff}\PY{o}{.}\PY{n}{total\PYZus{}seconds}\PY{p}{(}\PY{p}{)}\PY{p}{)}\PY{p}{)}
         \PY{n+nb}{print}\PY{p}{(}\PY{l+s+s1}{\PYZsq{}}\PY{l+s+s1}{On }\PY{l+s+s1}{\PYZsq{}} \PY{o}{+} \PY{n+nb}{str}\PY{p}{(}\PY{n}{bcss\PYZus{}kickoff}\PY{o}{.}\PY{n}{date}\PY{p}{(}\PY{p}{)} \PY{o}{+} \PY{l+m+mi}{2} \PY{o}{*} \PY{n}{time\PYZus{}since\PYZus{}kickoff}\PY{p}{)} 
               \PY{o}{+} \PY{l+s+s1}{\PYZsq{}}\PY{l+s+s1}{ we will be twice as old as we are now}\PY{l+s+s1}{\PYZsq{}}\PY{p}{)}
\end{Verbatim}


    \begin{Verbatim}[commandchars=\\\{\}]
It is 198292420 seconds since BCSS kickoff
On 2025-09-30 we will be twice as old as we are now

    \end{Verbatim}

    \hypertarget{pickle}{%
\subsubsection{Pickle}\label{pickle}}

Standard library module for saving to disk

    \begin{Verbatim}[commandchars=\\\{\}]
{\color{incolor}In [{\color{incolor}15}]:} \PY{k+kn}{import} \PY{n+nn}{pickle}
         
         \PY{c+c1}{\PYZsh{} save a string to a file:}
         \PY{n}{pickle}\PY{o}{.}\PY{n}{dump}\PY{p}{(}\PY{l+s+s1}{\PYZsq{}}\PY{l+s+s1}{A string that is saved in a file}\PY{l+s+s1}{\PYZsq{}}\PY{p}{,} \PY{n+nb}{open}\PY{p}{(}\PY{l+s+s1}{\PYZsq{}}\PY{l+s+s1}{my\PYZus{}pickle\PYZus{}file.p}\PY{l+s+s1}{\PYZsq{}}\PY{p}{,} \PY{l+s+s1}{\PYZsq{}}\PY{l+s+s1}{wb}\PY{l+s+s1}{\PYZsq{}}\PY{p}{)}\PY{p}{)}
         
         \PY{n}{file\PYZus{}content} \PY{o}{=} \PY{n}{pickle}\PY{o}{.}\PY{n}{load}\PY{p}{(}\PY{n+nb}{open}\PY{p}{(}\PY{l+s+s1}{\PYZsq{}}\PY{l+s+s1}{my\PYZus{}pickle\PYZus{}file.p}\PY{l+s+s1}{\PYZsq{}}\PY{p}{,} \PY{l+s+s1}{\PYZsq{}}\PY{l+s+s1}{rb}\PY{l+s+s1}{\PYZsq{}}\PY{p}{)}\PY{p}{)}
\end{Verbatim}


    \begin{Verbatim}[commandchars=\\\{\}]
{\color{incolor}In [{\color{incolor}16}]:} \PY{n+nb}{print}\PY{p}{(}\PY{n}{file\PYZus{}content}\PY{p}{)}
\end{Verbatim}


    \begin{Verbatim}[commandchars=\\\{\}]
A string that is saved in a file

    \end{Verbatim}

    Pickle is saved in binary format. Be careful, and dont use for things
that you will need for a long time

    \begin{Verbatim}[commandchars=\\\{\}]
{\color{incolor}In [{\color{incolor}17}]:} \PY{k}{class} \PY{n+nc}{Magnifier}\PY{p}{(}\PY{n+nb}{object}\PY{p}{)}\PY{p}{:}
             \PY{k}{def} \PY{n+nf}{\PYZus{}\PYZus{}init\PYZus{}\PYZus{}}\PY{p}{(}\PY{n+nb+bp}{self}\PY{p}{,} \PY{n}{x}\PY{p}{)}\PY{p}{:}
                 \PY{n+nb+bp}{self}\PY{o}{.}\PY{n}{x} \PY{o}{=} \PY{n}{x} \PY{o}{*} \PY{l+m+mi}{2}
         
             \PY{k}{def} \PY{n+nf}{write}\PY{p}{(}\PY{n+nb+bp}{self}\PY{p}{)}\PY{p}{:}
                 \PY{n+nb}{print}\PY{p}{(}\PY{n+nb+bp}{self}\PY{o}{.}\PY{n}{x}\PY{p}{)}
                 
         \PY{n}{b} \PY{o}{=} \PY{n}{Magnifier}\PY{p}{(}\PY{l+s+s1}{\PYZsq{}}\PY{l+s+s1}{hi }\PY{l+s+s1}{\PYZsq{}}\PY{p}{)}
         \PY{n}{b}\PY{o}{.}\PY{n}{write}\PY{p}{(}\PY{p}{)}
         
         \PY{n}{pickle}\PY{o}{.}\PY{n}{dump}\PY{p}{(}\PY{n}{b}\PY{p}{,} \PY{n+nb}{open}\PY{p}{(}\PY{l+s+s1}{\PYZsq{}}\PY{l+s+s1}{magnified\PYZus{}b.p}\PY{l+s+s1}{\PYZsq{}}\PY{p}{,} \PY{l+s+s1}{\PYZsq{}}\PY{l+s+s1}{wb}\PY{l+s+s1}{\PYZsq{}}\PY{p}{)}\PY{p}{)}
\end{Verbatim}


    \begin{Verbatim}[commandchars=\\\{\}]
hi hi 

    \end{Verbatim}

    \begin{Verbatim}[commandchars=\\\{\}]
{\color{incolor}In [{\color{incolor}18}]:} \PY{k}{class} \PY{n+nc}{Magnifier}\PY{p}{(}\PY{n+nb}{object}\PY{p}{)}\PY{p}{:} 
             \PY{k}{def} \PY{n+nf}{\PYZus{}\PYZus{}init\PYZus{}\PYZus{}}\PY{p}{(}\PY{n+nb+bp}{self}\PY{p}{,} \PY{n}{x}\PY{p}{)}\PY{p}{:}
                 \PY{n+nb+bp}{self}\PY{o}{.}\PY{n}{x} \PY{o}{=} \PY{n}{x} \PY{o}{*} \PY{l+m+mi}{4}
             
             \PY{k}{def} \PY{n+nf}{write}\PY{p}{(}\PY{n+nb+bp}{self}\PY{p}{)}\PY{p}{:}
                 \PY{n+nb}{print}\PY{p}{(}\PY{l+s+s1}{\PYZsq{}}\PY{l+s+s1}{Magnifier is now TWICE AS GOOD!}\PY{l+s+s1}{\PYZsq{}}\PY{p}{)}
                 
\end{Verbatim}


    \begin{Verbatim}[commandchars=\\\{\}]
{\color{incolor}In [{\color{incolor}19}]:} \PY{n}{b} \PY{o}{=} \PY{n}{pickle}\PY{o}{.}\PY{n}{load}\PY{p}{(}\PY{n+nb}{open}\PY{p}{(}\PY{l+s+s1}{\PYZsq{}}\PY{l+s+s1}{magnified\PYZus{}b.p}\PY{l+s+s1}{\PYZsq{}}\PY{p}{,} \PY{l+s+s1}{\PYZsq{}}\PY{l+s+s1}{rb}\PY{l+s+s1}{\PYZsq{}}\PY{p}{)}\PY{p}{)}
         \PY{n}{b}\PY{o}{.}\PY{n}{write}\PY{p}{(}\PY{p}{)}
         \PY{n+nb}{print}\PY{p}{(}\PY{n}{b}\PY{o}{.}\PY{n}{x}\PY{p}{)}
\end{Verbatim}


    \begin{Verbatim}[commandchars=\\\{\}]
Magnifier is now TWICE AS GOOD!
hi hi 

    \end{Verbatim}

    \hypertarget{numpy}{%
\subsubsection{Numpy }\label{numpy}}

Numpy is not shipped with Python. You need to download and install it
(but use anaconda/conda).

It is completely fundamental to all numerial applications. It is used in
all other modules that are discussed here

\emph{Books:} Guide to Numpy http://web.mit.edu/dvp/Public/numpybook.pdf
(Travis Oliphant) Scipy and Numpy (google it to find a free copy)

    The most important part of Numpy is the array:

    \begin{Verbatim}[commandchars=\\\{\}]
{\color{incolor}In [{\color{incolor}20}]:} \PY{k+kn}{import} \PY{n+nn}{numpy} \PY{k}{as} \PY{n+nn}{np}
         
         \PY{c+c1}{\PYZsh{} remember the list:}
         \PY{n}{my\PYZus{}list} \PY{o}{=} \PY{p}{[}\PY{l+m+mi}{1}\PY{p}{,} \PY{l+m+mi}{2}\PY{p}{,} \PY{l+m+mi}{3}\PY{p}{,} \PY{l+m+mi}{4}\PY{p}{]}
         \PY{n+nb}{print}\PY{p}{(}\PY{n}{my\PYZus{}list} \PY{o}{*} \PY{l+m+mi}{2}\PY{p}{)}
         
         \PY{c+c1}{\PYZsh{} convert the list to a Numpy array:}
         \PY{n}{x} \PY{o}{=} \PY{n}{np}\PY{o}{.}\PY{n}{array}\PY{p}{(}\PY{n}{my\PYZus{}list}\PY{p}{)}
         \PY{n+nb}{print}\PY{p}{(}\PY{n}{x}\PY{o}{*}\PY{l+m+mi}{2}\PY{p}{)}
\end{Verbatim}


    \begin{Verbatim}[commandchars=\\\{\}]
[1, 2, 3, 4, 1, 2, 3, 4]
[2 4 6 8]

    \end{Verbatim}

    \begin{Verbatim}[commandchars=\\\{\}]
{\color{incolor}In [{\color{incolor}21}]:} \PY{c+c1}{\PYZsh{} It has a specific data type:}
         \PY{n+nb}{print}\PY{p}{(}\PY{n}{x}\PY{o}{.}\PY{n}{dtype}\PY{p}{)}
\end{Verbatim}


    \begin{Verbatim}[commandchars=\\\{\}]
int64

    \end{Verbatim}

    \begin{Verbatim}[commandchars=\\\{\}]
{\color{incolor}In [{\color{incolor}22}]:} \PY{c+c1}{\PYZsh{} The data type is automatically changed when needed}
         \PY{n+nb}{print}\PY{p}{(}\PY{n}{np}\PY{o}{.}\PY{n}{sqrt}\PY{p}{(}\PY{n}{x}\PY{p}{)}\PY{o}{.}\PY{n}{dtype}\PY{p}{)}
\end{Verbatim}


    \begin{Verbatim}[commandchars=\\\{\}]
float64

    \end{Verbatim}

    \begin{Verbatim}[commandchars=\\\{\}]
{\color{incolor}In [{\color{incolor}23}]:} \PY{c+c1}{\PYZsh{} it is not as flexible as lists (for good reasons)}
         \PY{k}{try}\PY{p}{:}
             \PY{n}{x}\PY{p}{[}\PY{l+m+mi}{2}\PY{p}{]} \PY{o}{=} \PY{l+s+s1}{\PYZsq{}}\PY{l+s+s1}{a string}\PY{l+s+s1}{\PYZsq{}}
         \PY{k}{except}\PY{p}{:}
             \PY{n+nb}{print}\PY{p}{(}\PY{l+s+s2}{\PYZdq{}}\PY{l+s+s2}{x[2] = }\PY{l+s+s2}{\PYZsq{}}\PY{l+s+s2}{a string}\PY{l+s+s2}{\PYZsq{}}\PY{l+s+s2}{ did not work}\PY{l+s+se}{\PYZbs{}n}\PY{l+s+s2}{\PYZdq{}}\PY{p}{)}
             
         \PY{n+nb}{print}\PY{p}{(}\PY{n}{x}\PY{p}{)}
\end{Verbatim}


    \begin{Verbatim}[commandchars=\\\{\}]
x[2] = 'a string' did not work

[1 2 3 4]

    \end{Verbatim}

    \begin{Verbatim}[commandchars=\\\{\}]
{\color{incolor}In [{\color{incolor}24}]:} \PY{c+c1}{\PYZsh{}It can be multi\PYZhy{}dimensional:}
         \PY{n}{x} \PY{o}{=} \PY{n}{x}\PY{o}{.}\PY{n}{reshape}\PY{p}{(}\PY{p}{(}\PY{l+m+mi}{2}\PY{p}{,} \PY{l+m+mi}{2}\PY{p}{)}\PY{p}{)}
         \PY{n+nb}{print}\PY{p}{(}\PY{n}{x}\PY{p}{)}
\end{Verbatim}


    \begin{Verbatim}[commandchars=\\\{\}]
[[1 2]
 [3 4]]

    \end{Verbatim}

    \begin{Verbatim}[commandchars=\\\{\}]
{\color{incolor}In [{\color{incolor}25}]:} \PY{c+c1}{\PYZsh{} MULTI\PYZhy{}dimensional (adding 3 empty dimensions)}
         \PY{n}{x} \PY{o}{=} \PY{n}{x}\PY{o}{.}\PY{n}{reshape}\PY{p}{(}\PY{p}{(}\PY{l+m+mi}{2}\PY{p}{,} \PY{l+m+mi}{2}\PY{p}{,} \PY{l+m+mi}{1}\PY{p}{,} \PY{l+m+mi}{1}\PY{p}{,} \PY{l+m+mi}{1}\PY{p}{)}\PY{p}{)}
         \PY{n+nb}{print}\PY{p}{(}\PY{n}{x}\PY{o}{.}\PY{n}{ndim}\PY{p}{)}
\end{Verbatim}


    \begin{Verbatim}[commandchars=\\\{\}]
5

    \end{Verbatim}

    \begin{Verbatim}[commandchars=\\\{\}]
{\color{incolor}In [{\color{incolor}26}]:} \PY{c+c1}{\PYZsh{} the first dimension denotes rows, the second columns:}
         \PY{n}{x\PYZus{}row} \PY{o}{=} \PY{n}{x}\PY{o}{.}\PY{n}{reshape}\PY{p}{(}\PY{p}{(}\PY{l+m+mi}{1}\PY{p}{,} \PY{l+m+mi}{4}\PY{p}{)}\PY{p}{)}
         \PY{n+nb}{print}\PY{p}{(}\PY{n}{x\PYZus{}row}\PY{p}{)}
\end{Verbatim}


    \begin{Verbatim}[commandchars=\\\{\}]
[[1 2 3 4]]

    \end{Verbatim}

    \begin{Verbatim}[commandchars=\\\{\}]
{\color{incolor}In [{\color{incolor}27}]:} \PY{n}{x\PYZus{}column} \PY{o}{=} \PY{n}{x}\PY{o}{.}\PY{n}{reshape}\PY{p}{(}\PY{p}{(}\PY{l+m+mi}{4}\PY{p}{,} \PY{l+m+mi}{1}\PY{p}{)}\PY{p}{)}
         \PY{n+nb}{print}\PY{p}{(}\PY{n}{x\PYZus{}column}\PY{p}{)}
\end{Verbatim}


    \begin{Verbatim}[commandchars=\\\{\}]
[[1]
 [2]
 [3]
 [4]]

    \end{Verbatim}

    \begin{Verbatim}[commandchars=\\\{\}]
{\color{incolor}In [{\color{incolor}28}]:} \PY{c+c1}{\PYZsh{} Example of numpy array \PYZdq{}broadcasting\PYZdq{}:}
         \PY{n+nb}{print}\PY{p}{(}\PY{n}{x\PYZus{}row} \PY{o}{+} \PY{n}{x\PYZus{}column}\PY{p}{)}
\end{Verbatim}


    \begin{Verbatim}[commandchars=\\\{\}]
[[2 3 4 5]
 [3 4 5 6]
 [4 5 6 7]
 [5 6 7 8]]

    \end{Verbatim}

    \begin{Verbatim}[commandchars=\\\{\}]
{\color{incolor}In [{\color{incolor}29}]:} \PY{c+c1}{\PYZsh{} define 3 coordinate arrays (10 elements in each)}
         \PY{n}{x} \PY{o}{=} \PY{n}{np}\PY{o}{.}\PY{n}{random}\PY{o}{.}\PY{n}{random}\PY{p}{(}\PY{l+m+mi}{10}\PY{p}{)}
         \PY{n}{y} \PY{o}{=} \PY{n}{np}\PY{o}{.}\PY{n}{random}\PY{o}{.}\PY{n}{random}\PY{p}{(}\PY{l+m+mi}{10}\PY{p}{)}
         \PY{n}{z} \PY{o}{=} \PY{n}{np}\PY{o}{.}\PY{n}{random}\PY{o}{.}\PY{n}{random}\PY{p}{(}\PY{l+m+mi}{10}\PY{p}{)}
         
         \PY{c+c1}{\PYZsh{} combine to 10 3D vectors:}
         \PY{n}{r} \PY{o}{=} \PY{n}{np}\PY{o}{.}\PY{n}{vstack}\PY{p}{(}\PY{p}{(}\PY{n}{x}\PY{p}{,} \PY{n}{y}\PY{p}{,} \PY{n}{z}\PY{p}{)}\PY{p}{)}
         \PY{n+nb}{print}\PY{p}{(}\PY{n}{r}\PY{o}{.}\PY{n}{shape}\PY{p}{)}
\end{Verbatim}


    \begin{Verbatim}[commandchars=\\\{\}]
(3, 10)

    \end{Verbatim}

    \begin{Verbatim}[commandchars=\\\{\}]
{\color{incolor}In [{\color{incolor}30}]:} \PY{c+c1}{\PYZsh{} print single element of r:}
         \PY{n+nb}{print}\PY{p}{(}\PY{n}{r}\PY{p}{[}\PY{l+m+mi}{0}\PY{p}{,} \PY{l+m+mi}{9}\PY{p}{]}\PY{p}{)}
\end{Verbatim}


    \begin{Verbatim}[commandchars=\\\{\}]
0.05826540652401235

    \end{Verbatim}

    \begin{Verbatim}[commandchars=\\\{\}]
{\color{incolor}In [{\color{incolor}31}]:} \PY{c+c1}{\PYZsh{} print single row:}
         \PY{n+nb}{print}\PY{p}{(}\PY{n}{r}\PY{p}{[}\PY{l+m+mi}{0}\PY{p}{]}\PY{p}{)}
\end{Verbatim}


    \begin{Verbatim}[commandchars=\\\{\}]
[0.47947988 0.41797065 0.39095365 0.6604869  0.64901782 0.01312727
 0.65556763 0.87494596 0.21994291 0.05826541]

    \end{Verbatim}

    \begin{Verbatim}[commandchars=\\\{\}]
{\color{incolor}In [{\color{incolor}32}]:} \PY{c+c1}{\PYZsh{} the means of all columns:}
         \PY{n+nb}{print}\PY{p}{(}\PY{n}{np}\PY{o}{.}\PY{n}{mean}\PY{p}{(}\PY{n}{r}\PY{p}{,} \PY{n}{axis} \PY{o}{=} \PY{l+m+mi}{0}\PY{p}{)}\PY{p}{)}
\end{Verbatim}


    \begin{Verbatim}[commandchars=\\\{\}]
[0.48085679 0.72672689 0.5566112  0.43242048 0.46950691 0.55639486
 0.7943157  0.37082681 0.393878   0.25289113]

    \end{Verbatim}

    \begin{Verbatim}[commandchars=\\\{\}]
{\color{incolor}In [{\color{incolor}33}]:} \PY{c+c1}{\PYZsh{} the lengths of all columns:}
         \PY{n+nb}{print}\PY{p}{(}\PY{n}{np}\PY{o}{.}\PY{n}{linalg}\PY{o}{.}\PY{n}{norm}\PY{p}{(}\PY{n}{r}\PY{p}{,} \PY{n}{axis} \PY{o}{=} \PY{l+m+mi}{0}\PY{p}{)} \PY{p}{)}
\end{Verbatim}


    \begin{Verbatim}[commandchars=\\\{\}]
[0.96552471 1.32473776 1.07400757 0.91561962 0.9355609  1.17872637
 1.39682648 0.89739116 0.74073111 0.53080091]

    \end{Verbatim}

    \begin{Verbatim}[commandchars=\\\{\}]
{\color{incolor}In [{\color{incolor}34}]:} \PY{c+c1}{\PYZsh{} normalize the vectors:}
         \PY{n}{r} \PY{o}{=} \PY{n}{r} \PY{o}{/} \PY{n}{np}\PY{o}{.}\PY{n}{linalg}\PY{o}{.}\PY{n}{norm}\PY{p}{(}\PY{n}{r}\PY{p}{,} \PY{n}{axis} \PY{o}{=} \PY{l+m+mi}{0}\PY{p}{)}
         
         \PY{c+c1}{\PYZsh{} check that it worked:}
         \PY{n+nb}{print}\PY{p}{(}\PY{n}{np}\PY{o}{.}\PY{n}{linalg}\PY{o}{.}\PY{n}{norm}\PY{p}{(}\PY{n}{r}\PY{p}{,} \PY{n}{axis} \PY{o}{=} \PY{l+m+mi}{0}\PY{p}{)}\PY{p}{)}
\end{Verbatim}


    \begin{Verbatim}[commandchars=\\\{\}]
[1. 1. 1. 1. 1. 1. 1. 1. 1. 1.]

    \end{Verbatim}

    \begin{Verbatim}[commandchars=\\\{\}]
{\color{incolor}In [{\color{incolor}35}]:} \PY{c+c1}{\PYZsh{} Define a rotation matrix (90 degree rotation about z axis):}
         \PY{n}{R} \PY{o}{=} \PY{n}{np}\PY{o}{.}\PY{n}{array}\PY{p}{(}\PY{p}{[}\PY{p}{[}\PY{l+m+mi}{0}\PY{p}{,} \PY{o}{\PYZhy{}}\PY{l+m+mi}{1}\PY{p}{,} \PY{l+m+mi}{0}\PY{p}{]}\PY{p}{,} \PY{p}{[}\PY{l+m+mi}{1}\PY{p}{,} \PY{l+m+mi}{0}\PY{p}{,} \PY{l+m+mi}{0}\PY{p}{]}\PY{p}{,} \PY{p}{[}\PY{l+m+mi}{0}\PY{p}{,} \PY{l+m+mi}{0}\PY{p}{,} \PY{l+m+mi}{1}\PY{p}{]}\PY{p}{]}\PY{p}{)}
         
         \PY{c+c1}{\PYZsh{} Apply it to the vectors using the dot function:}
         \PY{n}{r\PYZus{}rotated} \PY{o}{=} \PY{n}{R}\PY{o}{.}\PY{n}{dot}\PY{p}{(}\PY{n}{r}\PY{p}{)}
\end{Verbatim}


    \hypertarget{sympy}{%
\subsubsection{Sympy }\label{sympy}}

Sympy is used for symbolic mathematics.

\emph{Resources:} Book: Instant Sympy Starter https://docs.sympy.org

    \begin{Verbatim}[commandchars=\\\{\}]
{\color{incolor}In [{\color{incolor}36}]:} \PY{k+kn}{import} \PY{n+nn}{sympy} \PY{k}{as} \PY{n+nn}{sp}
         \PY{n}{sp}\PY{o}{.}\PY{n}{init\PYZus{}printing}\PY{p}{(}\PY{p}{)} \PY{c+c1}{\PYZsh{} this line is to get nicely formatted output}
         
         \PY{n}{x} \PY{o}{=} \PY{n}{sp}\PY{o}{.}\PY{n}{symbols}\PY{p}{(}\PY{l+s+s1}{\PYZsq{}}\PY{l+s+s1}{x}\PY{l+s+s1}{\PYZsq{}}\PY{p}{)} 
         
         \PY{n}{sp}\PY{o}{.}\PY{n}{integrate}\PY{p}{(}\PY{l+m+mi}{1}\PY{o}{/}\PY{n}{sp}\PY{o}{.}\PY{n}{sin}\PY{p}{(}\PY{n}{x}\PY{p}{)}\PY{o}{*}\PY{o}{*}\PY{l+m+mi}{3}\PY{p}{)}
\end{Verbatim}

\texttt{\color{outcolor}Out[{\color{outcolor}36}]:}
    
    $$\frac{\log{\left (\cos{\left (x \right )} - 1 \right )}}{4} - \frac{\log{\left (\cos{\left (x \right )} + 1 \right )}}{4} + \frac{\cos{\left (x \right )}}{2 \cos^{2}{\left (x \right )} - 2}$$

    

    \hypertarget{matplotlib}{%
\subsubsection{Matplotlib }\label{matplotlib}}

Matplotlib is used for plotting.

The only sensible way to learn matplotlib is by examples:
https://matplotlib.org/gallery/index.html Stackoverflow has
\textasciitilde{}100 000 posts about matplotlib

    \begin{Verbatim}[commandchars=\\\{\}]
{\color{incolor}In [{\color{incolor}37}]:} \PY{k+kn}{import} \PY{n+nn}{matplotlib}\PY{n+nn}{.}\PY{n+nn}{pyplot} \PY{k}{as} \PY{n+nn}{plt}
         
         \PY{n}{fig} \PY{o}{=} \PY{n}{plt}\PY{o}{.}\PY{n}{figure}\PY{p}{(}\PY{p}{)}
         \PY{n}{ax} \PY{o}{=} \PY{n}{fig}\PY{o}{.}\PY{n}{add\PYZus{}subplot}\PY{p}{(}\PY{l+m+mi}{111}\PY{p}{)}
         
         \PY{n}{x} \PY{o}{=} \PY{n}{np}\PY{o}{.}\PY{n}{linspace}\PY{p}{(}\PY{l+m+mi}{0}\PY{p}{,} \PY{l+m+mi}{4} \PY{o}{*} \PY{n}{np}\PY{o}{.}\PY{n}{pi}\PY{p}{,} \PY{l+m+mi}{1000}\PY{p}{)}
         \PY{n}{ax}\PY{o}{.}\PY{n}{plot}\PY{p}{(}\PY{n}{x}\PY{p}{,} \PY{n}{np}\PY{o}{.}\PY{n}{cos}\PY{p}{(}\PY{n}{x}\PY{p}{)}\PY{p}{,} \PY{n}{label} \PY{o}{=} \PY{l+s+sa}{r}\PY{l+s+s1}{\PYZsq{}}\PY{l+s+s1}{\PYZdl{}}\PY{l+s+s1}{\PYZbs{}}\PY{l+s+s1}{cos(x)\PYZdl{}}\PY{l+s+s1}{\PYZsq{}}\PY{p}{)}
         \PY{n}{ax}\PY{o}{.}\PY{n}{legend}\PY{p}{(}\PY{n}{frameon} \PY{o}{=} \PY{k+kc}{False}\PY{p}{)}
         \PY{n}{ax}\PY{o}{.}\PY{n}{set\PYZus{}xlabel}\PY{p}{(}\PY{l+s+sa}{r}\PY{l+s+s1}{\PYZsq{}}\PY{l+s+s1}{\PYZdl{}}\PY{l+s+s1}{\PYZbs{}}\PY{l+s+s1}{theta\PYZdl{} [radians]}\PY{l+s+s1}{\PYZsq{}}\PY{p}{)}
         \PY{n}{plt}\PY{o}{.}\PY{n}{show}\PY{p}{(}\PY{p}{)}
\end{Verbatim}


    \begin{center}
    \adjustimage{max size={0.9\linewidth}{0.9\paperheight}}{output_53_0.png}
    \end{center}
    { \hspace*{\fill} \\}
    
    \hypertarget{scipy}{%
\subsubsection{SciPy }\label{scipy}}

Numpy provides the array class, but it is somewhat limited. SciPy is
kind of a Numpy expansion, a huge library of scientific software:
Optimization, linear algebra, integration, interpolation, special
functions, FFT, signal and image processing, ODE solvers, \ldots{}

\emph{Resources:} Numpy and Scipy (book) Documentation:
https://docs.scipy.org/doc/scipy/reference/ There are annual SciPy
conferences

    \begin{Verbatim}[commandchars=\\\{\}]
{\color{incolor}In [{\color{incolor}38}]:} \PY{k+kn}{from} \PY{n+nn}{scipy}\PY{n+nn}{.}\PY{n+nn}{special} \PY{k}{import} \PY{n}{sph\PYZus{}harm}
         
         \PY{c+c1}{\PYZsh{} make a grid of colatitude and longitude:}
         \PY{n}{colat}\PY{p}{,} \PY{n}{lon} \PY{o}{=} \PY{n}{np}\PY{o}{.}\PY{n}{meshgrid}\PY{p}{(}\PY{n}{np}\PY{o}{.}\PY{n}{linspace}\PY{p}{(}\PY{l+m+mi}{0}\PY{p}{,} \PY{n}{np}\PY{o}{.}\PY{n}{pi}\PY{p}{,} \PY{l+m+mi}{180}\PY{p}{)}\PY{p}{,} \PY{n}{np}\PY{o}{.}\PY{n}{linspace}\PY{p}{(}\PY{o}{\PYZhy{}}\PY{n}{np}\PY{o}{.}\PY{n}{pi} \PY{p}{,} \PY{n}{np}\PY{o}{.}\PY{n}{pi} \PY{p}{,} \PY{l+m+mi}{360}\PY{p}{)}\PY{p}{)}
         
         \PY{c+c1}{\PYZsh{} calculate spherical harmonics of degree 60 and order 40 at grid points: }
         \PY{n}{y} \PY{o}{=} \PY{n}{sph\PYZus{}harm}\PY{p}{(}\PY{l+m+mi}{40}\PY{p}{,} \PY{l+m+mi}{60}\PY{p}{,} \PY{n}{lon}\PY{p}{,} \PY{n}{colat}\PY{p}{)} 
\end{Verbatim}


    \begin{Verbatim}[commandchars=\\\{\}]
{\color{incolor}In [{\color{incolor}39}]:} \PY{c+c1}{\PYZsh{} plot the result on a globe using basemap:}
         \PY{k+kn}{from} \PY{n+nn}{mpl\PYZus{}toolkits}\PY{n+nn}{.}\PY{n+nn}{basemap} \PY{k}{import} \PY{n}{Basemap}
         \PY{n}{ax} \PY{o}{=} \PY{n}{plt}\PY{o}{.}\PY{n}{figure}\PY{p}{(}\PY{n}{figsize} \PY{o}{=} \PY{p}{(}\PY{l+m+mi}{8}\PY{p}{,} \PY{l+m+mi}{7}\PY{p}{)}\PY{p}{)}\PY{o}{.}\PY{n}{add\PYZus{}subplot}\PY{p}{(}\PY{l+m+mi}{111}\PY{p}{)}
         
         \PY{n}{m} \PY{o}{=} \PY{n}{Basemap}\PY{p}{(}\PY{n}{projection} \PY{o}{=} \PY{l+s+s1}{\PYZsq{}}\PY{l+s+s1}{ortho}\PY{l+s+s1}{\PYZsq{}}\PY{p}{,} \PY{n}{lon\PYZus{}0} \PY{o}{=} \PY{l+m+mi}{0}\PY{p}{,} \PY{n}{lat\PYZus{}0} \PY{o}{=} \PY{l+m+mi}{30}\PY{p}{,} \PY{n}{ax} \PY{o}{=} \PY{n}{ax}\PY{p}{)}
         \PY{n}{c} \PY{o}{=} \PY{n}{m}\PY{o}{.}\PY{n}{contourf}\PY{p}{(}\PY{n}{np}\PY{o}{.}\PY{n}{rad2deg}\PY{p}{(}\PY{n}{lon}\PY{p}{)}\PY{p}{,} \PY{l+m+mi}{90} \PY{o}{\PYZhy{}} \PY{n}{np}\PY{o}{.}\PY{n}{rad2deg}\PY{p}{(}\PY{n}{colat}\PY{p}{)}\PY{p}{,} \PY{n}{y}\PY{p}{,} \PY{n}{cmap} \PY{o}{=} \PY{n}{plt}\PY{o}{.}\PY{n}{cm}\PY{o}{.}\PY{n}{bwr}\PY{p}{,}
                        \PY{n}{levels} \PY{o}{=} \PY{n}{np}\PY{o}{.}\PY{n}{linspace}\PY{p}{(}\PY{o}{\PYZhy{}}\PY{l+m+mi}{1}\PY{p}{,} \PY{l+m+mi}{1}\PY{p}{,} \PY{l+m+mi}{22}\PY{p}{)}\PY{p}{,} \PY{n}{latlon} \PY{o}{=} \PY{k+kc}{True}\PY{p}{)}
         \PY{n}{plt}\PY{o}{.}\PY{n}{colorbar}\PY{p}{(}\PY{n}{c}\PY{p}{)}
         \PY{n}{m}\PY{o}{.}\PY{n}{drawparallels}\PY{p}{(}\PY{n}{np}\PY{o}{.}\PY{n}{r\PYZus{}}\PY{p}{[}\PY{o}{\PYZhy{}}\PY{l+m+mi}{60}\PY{p}{:}\PY{l+m+mi}{61}\PY{p}{:}\PY{l+m+mi}{30}\PY{p}{]}\PY{p}{)}
         \PY{n}{m}\PY{o}{.}\PY{n}{drawmeridians}\PY{p}{(}\PY{n}{np}\PY{o}{.}\PY{n}{r\PYZus{}}\PY{p}{[}\PY{o}{\PYZhy{}}\PY{l+m+mi}{180}\PY{p}{:}\PY{l+m+mi}{180}\PY{p}{:}\PY{l+m+mi}{60}\PY{p}{]}\PY{p}{)}
         \PY{n}{plt}\PY{o}{.}\PY{n}{show}\PY{p}{(}\PY{p}{)}
\end{Verbatim}


    \begin{center}
    \adjustimage{max size={0.9\linewidth}{0.9\paperheight}}{output_56_0.png}
    \end{center}
    { \hspace*{\fill} \\}
    
    \begin{Verbatim}[commandchars=\\\{\}]
{\color{incolor}In [{\color{incolor}40}]:} \PY{c+c1}{\PYZsh{} Scipy can also be used to read and write Matlab files:}
         \PY{k+kn}{from} \PY{n+nn}{scipy}\PY{n+nn}{.}\PY{n+nn}{io}\PY{n+nn}{.}\PY{n+nn}{matlab} \PY{k}{import} \PY{n}{loadmat}\PY{p}{,} \PY{n}{savemat}
         
         \PY{c+c1}{\PYZsh{} and read IDL files:}
         \PY{k+kn}{from} \PY{n+nn}{scipy}\PY{n+nn}{.}\PY{n+nn}{io}\PY{n+nn}{.}\PY{n+nn}{idl} \PY{k}{import} \PY{n}{readsav}
\end{Verbatim}


    \begin{Verbatim}[commandchars=\\\{\}]
{\color{incolor}In [{\color{incolor}41}]:} \PY{c+c1}{\PYZsh{} Scipy also provides access to BLAS and LAPACK functions:}
         \PY{k+kn}{from} \PY{n+nn}{scipy}\PY{n+nn}{.}\PY{n+nn}{linalg} \PY{k}{import} \PY{n}{blas}
         \PY{k+kn}{from} \PY{n+nn}{scipy}\PY{n+nn}{.}\PY{n+nn}{linalg} \PY{k}{import} \PY{n}{lapack}
         
         \PY{c+c1}{\PYZsh{} example: calculating c * A.T dot A, when c is a scalar and A is a complex matrix}
         \PY{n}{A} \PY{o}{=} \PY{n}{np}\PY{o}{.}\PY{n}{random}\PY{o}{.}\PY{n}{random}\PY{p}{(}\PY{p}{(}\PY{l+m+mi}{500}\PY{p}{,} \PY{l+m+mi}{500}\PY{p}{)}\PY{p}{)} \PY{o}{+} \PY{n}{np}\PY{o}{.}\PY{n}{random}\PY{o}{.}\PY{n}{random}\PY{p}{(}\PY{p}{(}\PY{l+m+mi}{500}\PY{p}{,} \PY{l+m+mi}{500}\PY{p}{)}\PY{p}{)} \PY{o}{*} \PY{l+m+mi}{1}\PY{n}{j}
         \PY{n}{ATA1} \PY{o}{=} \PY{n}{np}\PY{o}{.}\PY{n}{triu}\PY{p}{(} \PY{l+m+mi}{4} \PY{o}{*} \PY{n}{A}\PY{o}{.}\PY{n}{T}\PY{o}{.}\PY{n}{dot}\PY{p}{(}\PY{n}{A}\PY{p}{)}     \PY{p}{)}          
         \PY{n}{ATA2} \PY{o}{=} \PY{n}{np}\PY{o}{.}\PY{n}{triu}\PY{p}{(} \PY{n}{blas}\PY{o}{.}\PY{n}{csyrk}\PY{p}{(}\PY{l+m+mi}{4}\PY{p}{,} \PY{n}{A}\PY{p}{,} \PY{n}{trans} \PY{o}{=} \PY{l+m+mi}{1}\PY{p}{)} \PY{p}{)}
         
         \PY{c+c1}{\PYZsh{} make sure that the results are (almost) equal:}
         \PY{k}{assert} \PY{n}{np}\PY{o}{.}\PY{n}{all}\PY{p}{(}\PY{n}{np}\PY{o}{.}\PY{n}{isclose}\PY{p}{(} \PY{n}{np}\PY{o}{.}\PY{n}{abs}\PY{p}{(}\PY{n}{ATA1} \PY{o}{\PYZhy{}} \PY{n}{ATA2}\PY{p}{)}\PY{p}{,} \PY{l+m+mi}{0} \PY{p}{,} \PY{n}{atol} \PY{o}{=} \PY{l+m+mf}{1e\PYZhy{}3}\PY{p}{)}\PY{p}{)}
         
         \PY{c+c1}{\PYZsh{} look at the time difference }
         \PY{o}{\PYZpc{}}\PY{k}{timeit} 4 * A.T.dot(A) 
         \PY{o}{\PYZpc{}}\PY{k}{timeit} blas.csyrk(4, A, trans = 1) 
         \PY{c+c1}{\PYZsh{} (when the number of rows in A increases, the difference gets smaller)}
\end{Verbatim}


    \begin{Verbatim}[commandchars=\\\{\}]
4.91 ms ± 294 µs per loop (mean ± std. dev. of 7 runs, 100 loops each)
2.35 ms ± 123 µs per loop (mean ± std. dev. of 7 runs, 100 loops each)

    \end{Verbatim}

    \hypertarget{pandas}{%
\subsubsection{Pandas }\label{pandas}}

Pandas is a data analysis library. It is extremely useful for reading,
processing, combining data

\emph{Resources:} Python for Data Analysis (book - I have a copy)
Documentation: https://pandas.pydata.org/pandas-docs/stable/

    \begin{Verbatim}[commandchars=\\\{\}]
{\color{incolor}In [{\color{incolor}42}]:} \PY{k+kn}{import} \PY{n+nn}{pandas} \PY{k}{as} \PY{n+nn}{pd}
         
         \PY{c+c1}{\PYZsh{} read a file downloaded from supermag, and store it in a pandas DataFrame:}
         \PY{n}{data} \PY{o}{=} \PY{n}{pd}\PY{o}{.}\PY{n}{read\PYZus{}csv}\PY{p}{(}\PY{l+s+s1}{\PYZsq{}}\PY{l+s+s1}{supermagfile.csv}\PY{l+s+s1}{\PYZsq{}}\PY{p}{,} \PY{n}{index\PYZus{}col} \PY{o}{=} \PY{l+m+mi}{0}\PY{p}{,} \PY{n}{parse\PYZus{}dates} \PY{o}{=} \PY{k+kc}{True}\PY{p}{)}
         \PY{n}{data}\PY{o}{.}\PY{n}{index}\PY{o}{.}\PY{n}{name} \PY{o}{=} \PY{l+s+s1}{\PYZsq{}}\PY{l+s+s1}{date (UTC)}\PY{l+s+s1}{\PYZsq{}} 
         
         \PY{n}{data}\PY{p}{[}\PY{p}{[}\PY{l+s+s1}{\PYZsq{}}\PY{l+s+s1}{SML}\PY{l+s+s1}{\PYZsq{}}\PY{p}{,} \PY{l+s+s1}{\PYZsq{}}\PY{l+s+s1}{SMU}\PY{l+s+s1}{\PYZsq{}}\PY{p}{]}\PY{p}{]}\PY{o}{.}\PY{n}{plot}\PY{p}{(}\PY{p}{)} \PY{c+c1}{\PYZsh{} in the background, pandas uses matplotlib to do this}
\end{Verbatim}


\begin{Verbatim}[commandchars=\\\{\}]
{\color{outcolor}Out[{\color{outcolor}42}]:} <matplotlib.axes.\_subplots.AxesSubplot at 0xb211ca860>
\end{Verbatim}
            
    \begin{center}
    \adjustimage{max size={0.9\linewidth}{0.9\paperheight}}{output_60_1.png}
    \end{center}
    { \hspace*{\fill} \\}
    
    \begin{Verbatim}[commandchars=\\\{\}]
{\color{incolor}In [{\color{incolor}43}]:} \PY{c+c1}{\PYZsh{} resample to 1 Hz (from 1 min resolution) and interpolate in one line:}
         \PY{n}{hdsml} \PY{o}{=} \PY{n}{data}\PY{p}{[}\PY{l+s+s1}{\PYZsq{}}\PY{l+s+s1}{SML}\PY{l+s+s1}{\PYZsq{}}\PY{p}{]}\PY{o}{.}\PY{n}{resample}\PY{p}{(}\PY{l+s+s1}{\PYZsq{}}\PY{l+s+s1}{1S}\PY{l+s+s1}{\PYZsq{}}\PY{p}{)}\PY{o}{.}\PY{n}{interpolate}\PY{p}{(}\PY{p}{)}
         
         \PY{c+c1}{\PYZsh{} easily combine different time series as long as there are matching indices}
         \PY{n}{smu} \PY{o}{=} \PY{n}{data}\PY{p}{[}\PY{l+s+s1}{\PYZsq{}}\PY{l+s+s1}{SMU}\PY{l+s+s1}{\PYZsq{}}\PY{p}{]}
         
         \PY{c+c1}{\PYZsh{} combine smu (1 min) and hdsml (1 Hz)}
         \PY{n}{ratio} \PY{o}{=} \PY{n}{smu} \PY{o}{/} \PY{n}{hdsml}
         
         \PY{c+c1}{\PYZsh{} there will be NaNs where hdsml has data and smu does not}
         \PY{n+nb}{print}\PY{p}{(}\PY{l+s+s1}{\PYZsq{}}\PY{l+s+si}{\PYZpc{}.1f}\PY{l+s+si}{\PYZpc{}\PYZpc{}}\PY{l+s+s1}{ of the elements in ratio are NaNs (59/60 = }\PY{l+s+si}{\PYZpc{}.3f}\PY{l+s+s1}{)}\PY{l+s+s1}{\PYZsq{}} \PY{o}{\PYZpc{}}
               \PY{p}{(}\PY{p}{(}\PY{l+m+mi}{1} \PY{o}{\PYZhy{}} \PY{n+nb}{len}\PY{p}{(}\PY{n}{ratio}\PY{o}{.}\PY{n}{dropna}\PY{p}{(}\PY{p}{)}\PY{p}{)} \PY{o}{/} \PY{n+nb}{len}\PY{p}{(}\PY{n}{ratio}\PY{p}{)} \PY{p}{)} \PY{o}{*} \PY{l+m+mi}{100}\PY{p}{,} \PY{l+m+mi}{59}\PY{o}{/}\PY{l+m+mi}{60}\PY{p}{)}\PY{p}{)}
\end{Verbatim}


    \begin{Verbatim}[commandchars=\\\{\}]
98.3\% of the elements in ratio are NaNs (59/60 = 0.983)

    \end{Verbatim}

    \begin{Verbatim}[commandchars=\\\{\}]
{\color{incolor}In [{\color{incolor}44}]:} \PY{c+c1}{\PYZsh{} easily select a subset of one day:}
         \PY{n}{hdsml}\PY{p}{[}\PY{l+s+s1}{\PYZsq{}}\PY{l+s+s1}{2001\PYZhy{}11\PYZhy{}24}\PY{l+s+s1}{\PYZsq{}}\PY{p}{]}\PY{o}{.}\PY{n}{plot}\PY{p}{(}\PY{p}{)}
         \PY{n}{plt}\PY{o}{.}\PY{n}{show}\PY{p}{(}\PY{p}{)}
\end{Verbatim}


    \begin{center}
    \adjustimage{max size={0.9\linewidth}{0.9\paperheight}}{output_62_0.png}
    \end{center}
    { \hspace*{\fill} \\}
    
    \begin{Verbatim}[commandchars=\\\{\}]
{\color{incolor}In [{\color{incolor}45}]:} \PY{c+c1}{\PYZsh{} analyze how SML amd SMU covary:}
         \PY{c+c1}{\PYZsh{} pd.qcut(data[\PYZsq{}SMU\PYZsq{}], 20) creates 20 groups / bins, defined by}
         \PY{c+c1}{\PYZsh{} SMU quantiles. In each of these bins, we calculate the median of SML}
         \PY{n}{sml\PYZus{}vs\PYZus{}smu} \PY{o}{=} \PY{n}{data}\PY{p}{[}\PY{l+s+s1}{\PYZsq{}}\PY{l+s+s1}{SML}\PY{l+s+s1}{\PYZsq{}}\PY{p}{]}\PY{o}{.}\PY{n}{groupby}\PY{p}{(} \PY{n}{pd}\PY{o}{.}\PY{n}{qcut}\PY{p}{(}\PY{n}{data}\PY{p}{[}\PY{l+s+s1}{\PYZsq{}}\PY{l+s+s1}{SMU}\PY{l+s+s1}{\PYZsq{}}\PY{p}{]}\PY{p}{,} \PY{l+m+mi}{20}\PY{p}{)} \PY{p}{)}\PY{o}{.}\PY{n}{median}\PY{p}{(}\PY{p}{)}
         
         \PY{c+c1}{\PYZsh{} print the first and last elements of the index}
         \PY{n+nb}{print}\PY{p}{(}\PY{n}{sml\PYZus{}vs\PYZus{}smu}\PY{o}{.}\PY{n}{index}\PY{p}{[}\PY{l+m+mi}{0}\PY{p}{]}\PY{p}{,} \PY{n}{sml\PYZus{}vs\PYZus{}smu}\PY{o}{.}\PY{n}{index}\PY{p}{[}\PY{o}{\PYZhy{}}\PY{l+m+mi}{1}\PY{p}{]}\PY{p}{)}
         
         \PY{c+c1}{\PYZsh{} make an array of the mid points of the smu intervals:}
         \PY{n}{smu} \PY{o}{=} \PY{p}{[}\PY{n}{x}\PY{o}{.}\PY{n}{mid} \PY{k}{for} \PY{n}{x} \PY{o+ow}{in} \PY{n}{sml\PYZus{}vs\PYZus{}smu}\PY{o}{.}\PY{n}{index}\PY{p}{]}
         
         \PY{n}{plt}\PY{o}{.}\PY{n}{semilogx}\PY{p}{(}\PY{n}{smu}\PY{p}{,} \PY{n}{sml\PYZus{}vs\PYZus{}smu}\PY{o}{.}\PY{n}{values}\PY{p}{,} \PY{n}{marker} \PY{o}{=} \PY{l+s+s1}{\PYZsq{}}\PY{l+s+s1}{o}\PY{l+s+s1}{\PYZsq{}}\PY{p}{)}
         \PY{n}{plt}\PY{o}{.}\PY{n}{xlabel}\PY{p}{(}\PY{l+s+s1}{\PYZsq{}}\PY{l+s+s1}{SMU}\PY{l+s+s1}{\PYZsq{}}\PY{p}{)}
         \PY{n}{plt}\PY{o}{.}\PY{n}{ylabel}\PY{p}{(}\PY{l+s+s1}{\PYZsq{}}\PY{l+s+s1}{SML}\PY{l+s+s1}{\PYZsq{}}\PY{p}{)}
         \PY{n}{plt}\PY{o}{.}\PY{n}{show}\PY{p}{(}\PY{p}{)}
\end{Verbatim}


    \begin{Verbatim}[commandchars=\\\{\}]
(-14.001, 20.0] (351.05, 1544.0]

    \end{Verbatim}

    \begin{center}
    \adjustimage{max size={0.9\linewidth}{0.9\paperheight}}{output_63_1.png}
    \end{center}
    { \hspace*{\fill} \\}
    
    \hypertarget{other-useful-libraries}{%
\subsubsection{Other useful libraries}\label{other-useful-libraries}}

Dask \(-\) For big data / parallel computing (https://dask.org/)
scikit-learn / sklearn: https://scikit-learn.org/stable/ \(-\) for
machine learning ++ tensorflow \(-\) also for machine learning,
developed by Google (https://www.tensorflow.org/) seaborn \(-\) for
making nice-looking plots quickly, built on matplotlib
(https://seaborn.pydata.org/) PIL \(-\) for image processing (Scipy also
has many useful image processing tools) lmfit \(-\) For non-linear least
squares with constraints (https://lmfit.github.io/lmfit-py/)

    \hypertarget{space-physics-modules}{%
\subsubsection{Space physics modules }\label{space-physics-modules}}

There's a ton of modules that have specific uses in space physics. A
nice overview is provided by \emph{Burrell, A. G., Halford, A.,
Klenzing, J., Stoneback, R. A., Morley, S. K., Annex, A. M., et al. (
2018). Snakes on a spaceship---An overview of Python in heliophysics.
Journal of Geophysical Research: Space Physics, 123, 10,384-- 10,402.
https://doi.org/10.1029/2018JA025877}

The article also presents a framework for community development, similar
to astronomy's AstroPy

I will present examples from different libraries providing access to
models originally written in fortran, coordinate conversion, and for
working with CDF (Common Data Format) files

Finally, I will show some examples from Pytt aka Pyrkeland

None of these modules are included with anaconda. You need to download
and install, and this can be tricky (especially on Windows)

    \hypertarget{cdf-common-data-format}{%
\subsubsection{CDF (Common Data Format)}\label{cdf-common-data-format}}

File format used extensively in space physics (binary)

\emph{SpacePy}, \emph{pysatCDF}, and \emph{CDFlib} are all able to load
CDF files. I recommend CDFlib (https://github.com/MAVENSDC/cdflib). Here
is an example:

    \begin{Verbatim}[commandchars=\\\{\}]
{\color{incolor}In [{\color{incolor}46}]:} \PY{k+kn}{import} \PY{n+nn}{cdflib}
         
         \PY{c+c1}{\PYZsh{} load a datafile containing 50Hz magnetic field measurements from Swarm}
         \PY{c+c1}{\PYZsh{} downloaded from swarm\PYZhy{}diss.eo.esa.int}
         \PY{n}{swarm\PYZus{}B\PYZus{}data} \PY{o}{=} \PY{n}{cdflib}\PY{o}{.}\PY{n}{CDF}\PY{p}{(}\PY{l+s+s1}{\PYZsq{}}\PY{l+s+s1}{./SW\PYZus{}OPER\PYZus{}MAGA\PYZus{}HR\PYZus{}1B\PYZus{}20150328T000000\PYZus{}20150328T235959\PYZus{}0505\PYZus{}MDR\PYZus{}MAG\PYZus{}HR.cdf}\PY{l+s+s1}{\PYZsq{}}\PY{p}{)}
         \PY{n}{B\PYZus{}vectors} \PY{o}{=} \PY{n}{swarm\PYZus{}B\PYZus{}data}\PY{o}{.}\PY{n}{varget}\PY{p}{(}\PY{l+s+s1}{\PYZsq{}}\PY{l+s+s1}{B\PYZus{}NEC}\PY{l+s+s1}{\PYZsq{}}\PY{p}{)} \PY{c+c1}{\PYZsh{} (has shape N x 3)}
         \PY{n}{B\PYZus{}east} \PY{o}{=} \PY{n}{B\PYZus{}vectors}\PY{p}{[}\PY{p}{:}\PY{p}{,} \PY{l+m+mi}{1}\PY{p}{]}\PY{o}{.}\PY{n}{flatten}\PY{p}{(}\PY{p}{)} \PY{c+c1}{\PYZsh{} eastward component}
         
         \PY{c+c1}{\PYZsh{} plot the time derivative (first difference) of the last 1000 elements}
         \PY{n}{plt}\PY{o}{.}\PY{n}{plot}\PY{p}{(}\PY{n}{np}\PY{o}{.}\PY{n}{diff}\PY{p}{(} \PY{n}{B\PYZus{}east}\PY{p}{[}\PY{o}{\PYZhy{}}\PY{l+m+mi}{1000}\PY{p}{:}\PY{p}{]} \PY{p}{)}\PY{p}{)}
\end{Verbatim}


\begin{Verbatim}[commandchars=\\\{\}]
{\color{outcolor}Out[{\color{outcolor}46}]:} [<matplotlib.lines.Line2D at 0xb219e5470>]
\end{Verbatim}
            
    \begin{center}
    \adjustimage{max size={0.9\linewidth}{0.9\paperheight}}{output_67_1.png}
    \end{center}
    { \hspace*{\fill} \\}
    
    \hypertarget{international-geomagnetic-reference-field-igrf}{%
\subsubsection{International Geomagnetic Reference Field
(IGRF)}\label{international-geomagnetic-reference-field-igrf}}

A standard model of Earth's magnetic field.

I have written a pure Python implementation of this, but it \textbf{may}
be better to use something more public. The Snakes on a Spaceship paper
mentions \emph{pyglow}. It is fairly easy to install if you follow the
instructions (https://github.com/timduly4/pyglow)

    \begin{Verbatim}[commandchars=\\\{\}]
{\color{incolor}In [{\color{incolor}47}]:} \PY{k+kn}{import} \PY{n+nn}{pyglow}
         
         \PY{c+c1}{\PYZsh{} Parameters:}
         \PY{n}{colat} \PY{o}{=} \PY{l+m+mf}{29.776}
         \PY{n}{lon}   \PY{o}{=} \PY{l+m+mf}{5.33}
         \PY{n}{isv}   \PY{o}{=} \PY{l+m+mi}{0} \PY{c+c1}{\PYZsh{} 0 if you want the main field, 1 if you want the secular variation}
         \PY{n}{itype} \PY{o}{=} \PY{l+m+mi}{2} \PY{c+c1}{\PYZsh{} 1 if input coords are geodetic, 2 if geocentric}
         \PY{n}{r}     \PY{o}{=} \PY{l+m+mf}{6362.006324680832}
         \PY{n}{date}  \PY{o}{=} \PY{l+m+mf}{2019.1}
         
         \PY{n}{B} \PY{o}{=} \PY{n}{pyglow}\PY{o}{.}\PY{n}{igrf}\PY{o}{.}\PY{n}{igrf12}\PY{p}{(}\PY{n}{isv}\PY{p}{,} \PY{n}{date}\PY{p}{,} \PY{n}{itype}\PY{p}{,} \PY{n}{r}\PY{p}{,} \PY{n}{colat}\PY{p}{,} \PY{n}{lon}\PY{p}{)}
         \PY{n+nb}{print}\PY{p}{(}\PY{l+s+s1}{\PYZsq{}}\PY{l+s+s1}{north: }\PY{l+s+si}{\PYZpc{}.1f}\PY{l+s+s1}{ nT, east: }\PY{l+s+si}{\PYZpc{}.1f}\PY{l+s+s1}{, vertical: }\PY{l+s+si}{\PYZpc{}.1f}\PY{l+s+s1}{ nT, total: }\PY{l+s+si}{\PYZpc{}.1f}\PY{l+s+s1}{ nT}\PY{l+s+s1}{\PYZsq{}} \PY{o}{\PYZpc{}} \PY{n}{B}\PY{p}{)}
\end{Verbatim}


    \begin{Verbatim}[commandchars=\\\{\}]
north: 14871.7 nT, east: 313.6, vertical: 48925.3 nT, total: 51136.6 nT

    \end{Verbatim}

    With the igrf12 function the input has to be scalar. To make an array
version of this, one can use \textbf{np.frompyfunc}, which takes an
arbitrary function and returns a function that works well with numpy
arrays

    \hypertarget{chaos-model}{%
\subsubsection{CHAOS model}\label{chaos-model}}

The DTU main field model. Much more detailed than IGRF. They have
recently published Python forward code, called \emph{chaosmagpy}

    \begin{Verbatim}[commandchars=\\\{\}]
{\color{incolor}In [{\color{incolor}48}]:} \PY{k+kn}{from} \PY{n+nn}{chaosmagpy} \PY{k}{import} \PY{n}{load\PYZus{}CHAOS\PYZus{}matfile}
         
         \PY{n}{chaosmodel} \PY{o}{=} \PY{n}{load\PYZus{}CHAOS\PYZus{}matfile}\PY{p}{(}\PY{l+s+s1}{\PYZsq{}}\PY{l+s+s1}{./CHAOS\PYZhy{}6\PYZhy{}x8.mat}\PY{l+s+s1}{\PYZsq{}}\PY{p}{)}
         
         \PY{n}{mjd2000} \PY{o}{=} \PY{p}{(}\PY{n}{date} \PY{o}{\PYZhy{}} \PY{l+m+mi}{2000}\PY{p}{)} \PY{o}{*} \PY{l+m+mf}{365.25} 
         \PY{n}{B} \PY{o}{=} \PY{n}{chaosmodel}\PY{p}{(}\PY{n}{mjd2000}\PY{p}{,} \PY{n}{r}\PY{p}{,} \PY{n}{colat}\PY{p}{,} \PY{n}{lon}\PY{p}{)} \PY{c+c1}{\PYZsh{} return components are in r, theta, phi directions}
         
         \PY{n+nb}{print}\PY{p}{(}\PY{l+s+s1}{\PYZsq{}}\PY{l+s+s1}{north: }\PY{l+s+si}{\PYZpc{}.1f}\PY{l+s+s1}{ nT, east: }\PY{l+s+si}{\PYZpc{}.1f}\PY{l+s+s1}{, vertical: }\PY{l+s+si}{\PYZpc{}.1f}\PY{l+s+s1}{ nT}\PY{l+s+s1}{\PYZsq{}} \PY{o}{\PYZpc{}} \PY{p}{(}\PY{o}{\PYZhy{}}\PY{n}{B}\PY{p}{[}\PY{l+m+mi}{1}\PY{p}{]}\PY{p}{,} \PY{n}{B}\PY{p}{[}\PY{l+m+mi}{2}\PY{p}{]}\PY{p}{,} \PY{o}{\PYZhy{}}\PY{n}{B}\PY{p}{[}\PY{l+m+mi}{0}\PY{p}{]}\PY{p}{)}\PY{p}{)}
\end{Verbatim}


    \begin{Verbatim}[commandchars=\\\{\}]
Computing time-dependent internal field up to degree 20.
Finished in 0.0109999 seconds.
Computing static internal (i.e. small-scale crustal) field up to degree 85.
Finished in 0.173342 seconds.
Computing GSM field up to degree 2.
Finished in 0.00386203 seconds.
Computing SM field up to degree 2.
Finished in 0.298668 seconds.
north: 14824.6 nT, east: 347.2, vertical: 48976.1 nT

    \end{Verbatim}

    \hypertarget{msis-nrlmsise}{%
\subsubsection{MSIS / NRLMSISE}\label{msis-nrlmsise}}

Neutral Atmosphere Empirical Model from the surface to lower exosphere -
describes neutral densities and temperatures above 100 km

pyglow has a wrapper for this model also.
(https://github.com/timduly4/pyglow)

    \begin{Verbatim}[commandchars=\\\{\}]
{\color{incolor}In [{\color{incolor}49}]:} \PY{c+c1}{\PYZsh{} set some input parameters}
         \PY{n}{iyd} \PY{o}{=} \PY{l+m+mi}{19170} \PY{c+c1}{\PYZsh{} YYDDD (year \PYZhy{} day\PYZus{}of\PYZus{}year \PYZhy{} 19 June 2019 in this case)}
         \PY{n}{sec} \PY{o}{=} \PY{l+m+mi}{10} \PY{o}{*} \PY{l+m+mi}{60}\PY{o}{*}\PY{o}{*}\PY{l+m+mi}{1} \PY{c+c1}{\PYZsh{} 10 UT in seconds}
         \PY{n}{alt} \PY{o}{=} \PY{l+m+mi}{100} \PY{c+c1}{\PYZsh{} altitude in km}
         \PY{n}{lat}\PY{p}{,} \PY{n}{lon} \PY{o}{=} \PY{l+m+mf}{60.389}\PY{p}{,} \PY{l+m+mf}{5.33}
         \PY{n}{stl} \PY{o}{=} \PY{n}{sec}\PY{o}{/}\PY{l+m+mi}{3600} \PY{o}{+} \PY{n}{lon}\PY{o}{/}\PY{l+m+mi}{15} \PY{c+c1}{\PYZsh{} local apparent solar time (hrs) \PYZhy{} approximately}
         \PY{n}{f107a} \PY{o}{=} \PY{l+m+mi}{70} \PY{c+c1}{\PYZsh{} 80 day average f107 index centered on the day}
         \PY{n}{f107} \PY{o}{=} \PY{l+m+mi}{70}  \PY{c+c1}{\PYZsh{} f107 index yesterday}
         \PY{n}{ap}   \PY{o}{=} \PY{p}{[}\PY{l+m+mi}{67}\PY{p}{]}\PY{o}{*}\PY{l+m+mi}{7}  \PY{c+c1}{\PYZsh{} ap index \PYZhy{} daily + a set of previous values}
         \PY{n}{mass} \PY{o}{=} \PY{l+m+mi}{48}  \PY{c+c1}{\PYZsh{} not sure about this, but 48 is supposed to give all outputs}
         
         \PY{n}{d}\PY{p}{,} \PY{n}{t} \PY{o}{=} \PY{n}{pyglow}\PY{o}{.}\PY{n}{msis}\PY{o}{.}\PY{n}{msis00}\PY{p}{(}\PY{n}{iyd}\PY{p}{,}\PY{n}{sec}\PY{p}{,}\PY{n}{alt}\PY{p}{,}\PY{n}{lat}\PY{p}{,}\PY{n}{lon}\PY{p}{,}\PY{n}{stl}\PY{p}{,}\PY{n}{f107a}\PY{p}{,}\PY{n}{f107}\PY{p}{,}\PY{n}{ap}\PY{p}{,}\PY{n}{mass}\PY{p}{)}
         
         \PY{n}{plt}\PY{o}{.}\PY{n}{bar}\PY{p}{(}\PY{n}{pyglow}\PY{o}{.}\PY{n}{msis}\PY{o}{.}\PY{n}{CONSTITUENTS}\PY{p}{[}\PY{p}{:}\PY{o}{\PYZhy{}}\PY{l+m+mi}{1}\PY{p}{]}\PY{p}{,} \PY{n}{d}\PY{p}{[}\PY{n}{np}\PY{o}{.}\PY{n}{arange}\PY{p}{(}\PY{n}{d}\PY{o}{.}\PY{n}{size}\PY{p}{)} \PY{o}{!=} \PY{l+m+mi}{5}\PY{p}{]}\PY{p}{[}\PY{p}{:}\PY{o}{\PYZhy{}}\PY{l+m+mi}{1}\PY{p}{]}\PY{p}{)}
         \PY{n}{plt}\PY{o}{.}\PY{n}{yscale}\PY{p}{(}\PY{l+s+s1}{\PYZsq{}}\PY{l+s+s1}{log}\PY{l+s+s1}{\PYZsq{}}\PY{p}{)}
         \PY{n}{plt}\PY{o}{.}\PY{n}{ylabel}\PY{p}{(}\PY{l+s+s1}{\PYZsq{}}\PY{l+s+s1}{density in cm\PYZdl{}\PYZca{}}\PY{l+s+s1}{\PYZob{}}\PY{l+s+s1}{\PYZhy{}3\PYZcb{}\PYZdl{}}\PY{l+s+s1}{\PYZsq{}}\PY{p}{,} \PY{n}{size} \PY{o}{=} \PY{l+m+mi}{14}\PY{p}{)}
         \PY{n+nb}{print}\PY{p}{(}\PY{l+s+s1}{\PYZsq{}}\PY{l+s+s1}{Temperature at }\PY{l+s+si}{\PYZpc{}s}\PY{l+s+s1}{ km is }\PY{l+s+si}{\PYZpc{}.1f}\PY{l+s+s1}{ K}\PY{l+s+s1}{\PYZsq{}} \PY{o}{\PYZpc{}} \PY{p}{(}\PY{n}{alt}\PY{p}{,} \PY{n}{t}\PY{p}{[}\PY{l+m+mi}{0}\PY{p}{]}\PY{p}{)}\PY{p}{)}
\end{Verbatim}


    \begin{Verbatim}[commandchars=\\\{\}]
Temperature at 100 km is 1027.3 K

    \end{Verbatim}

    \begin{center}
    \adjustimage{max size={0.9\linewidth}{0.9\paperheight}}{output_74_1.png}
    \end{center}
    { \hspace*{\fill} \\}
    
    \hypertarget{hwm-horizontal-wind-model}{%
\subsubsection{HWM (Horizontal Wind
Model)}\label{hwm-horizontal-wind-model}}

Empirical model of horizontal neutral winds in the thermosphere. Python
wrapper is included in \emph{pyglow}

    \begin{Verbatim}[commandchars=\\\{\}]
{\color{incolor}In [{\color{incolor}50}]:} \PY{k+kn}{from} \PY{n+nn}{pyglow} \PY{k}{import} \PY{n}{hwm}
         
         \PY{c+c1}{\PYZsh{} Importing hwm works, but it crashes when I run it}
\end{Verbatim}


    \hypertarget{iri-international-reference-ionosphere}{%
\subsubsection{IRI (International Reference
Ionosphere)}\label{iri-international-reference-ionosphere}}

Empirical model of the ionosphere. \emph{pyglow} provides a Python
wrapper (https://github.com/timduly4/pyglow)

    \begin{Verbatim}[commandchars=\\\{\}]
{\color{incolor}In [{\color{incolor}51}]:} \PY{n}{alts} \PY{o}{=} \PY{n}{np}\PY{o}{.}\PY{n}{linspace}\PY{p}{(}\PY{l+m+mf}{100.}\PY{p}{,} \PY{l+m+mf}{500.}\PY{p}{,} \PY{l+m+mi}{101}\PY{p}{)}
         \PY{n}{dn} \PY{o}{=} \PY{n}{datetime}\PY{p}{(}\PY{l+m+mi}{2015}\PY{p}{,} \PY{l+m+mi}{6}\PY{p}{,} \PY{l+m+mi}{19}\PY{p}{,} \PY{l+m+mi}{10}\PY{p}{,} \PY{l+m+mi}{30}\PY{p}{)}
         
         \PY{n}{ne} \PY{o}{=} \PY{p}{[}\PY{p}{]}
         \PY{k}{for} \PY{n}{alt} \PY{o+ow}{in} \PY{n}{alts}\PY{p}{:} \PY{c+c1}{\PYZsh{} loop through altitudes}
             \PY{n}{pt} \PY{o}{=} \PY{n}{pyglow}\PY{o}{.}\PY{n}{Point}\PY{p}{(}\PY{n}{dn}\PY{p}{,} \PY{n}{lat}\PY{p}{,} \PY{n}{lon}\PY{p}{,} \PY{n}{alt}\PY{p}{)}
             \PY{n}{pt}\PY{o}{.}\PY{n}{run\PYZus{}iri}\PY{p}{(}\PY{p}{)} 
             \PY{n}{ne}\PY{o}{.}\PY{n}{append}\PY{p}{(}\PY{n}{pt}\PY{o}{.}\PY{n}{ne}\PY{p}{)} \PY{c+c1}{\PYZsh{} append() is a member function in the list class}
         
         \PY{n}{plt}\PY{o}{.}\PY{n}{semilogx}\PY{p}{(}\PY{n}{ne}\PY{p}{,} \PY{n}{alts}\PY{p}{,} \PY{l+s+s1}{\PYZsq{}}\PY{l+s+s1}{b\PYZhy{}}\PY{l+s+s1}{\PYZsq{}}\PY{p}{)} 
         \PY{n}{plt}\PY{o}{.}\PY{n}{xlabel}\PY{p}{(}\PY{l+s+sa}{r}\PY{l+s+s1}{\PYZsq{}}\PY{l+s+s1}{\PYZdl{}n\PYZus{}e\PYZdl{} [cm\PYZdl{}\PYZca{}}\PY{l+s+s1}{\PYZob{}}\PY{l+s+s1}{\PYZhy{}3\PYZcb{}\PYZdl{}]}\PY{l+s+s1}{\PYZsq{}}\PY{p}{,} \PY{n}{size} \PY{o}{=} \PY{l+m+mi}{14}\PY{p}{)}
         \PY{n}{plt}\PY{o}{.}\PY{n}{ylabel}\PY{p}{(}\PY{l+s+s1}{\PYZsq{}}\PY{l+s+s1}{Altitude [km]}\PY{l+s+s1}{\PYZsq{}}\PY{p}{,} \PY{n}{size} \PY{o}{=} \PY{l+m+mi}{14}\PY{p}{)}
         \PY{n}{plt}\PY{o}{.}\PY{n}{title}\PY{p}{(}\PY{l+s+sa}{r}\PY{l+s+s1}{\PYZsq{}}\PY{l+s+si}{\PYZpc{}s}\PY{l+s+s1}{ UT, lat=}\PY{l+s+si}{\PYZpc{}3.1f}\PY{l+s+s1}{\PYZdl{}\PYZca{}}\PY{l+s+s1}{\PYZbs{}}\PY{l+s+s1}{circ\PYZdl{}, lon=}\PY{l+s+si}{\PYZpc{}3.1f}\PY{l+s+s1}{\PYZdl{}\PYZca{}}\PY{l+s+s1}{\PYZbs{}}\PY{l+s+s1}{circ\PYZdl{}}\PY{l+s+s1}{\PYZsq{}} \PY{o}{\PYZpc{}} \PY{p}{(}\PY{n+nb}{str}\PY{p}{(}\PY{n}{dn}\PY{p}{)}\PY{p}{,} \PY{n}{lat}\PY{p}{,} \PY{n}{lon}\PY{p}{)}\PY{p}{,} \PY{n}{size} \PY{o}{=} \PY{l+m+mi}{14}\PY{p}{)}
         \PY{n}{plt}\PY{o}{.}\PY{n}{show}\PY{p}{(}\PY{p}{)}
\end{Verbatim}


    \begin{center}
    \adjustimage{max size={0.9\linewidth}{0.9\paperheight}}{output_78_0.png}
    \end{center}
    { \hspace*{\fill} \\}
    
    \hypertarget{tsyganenko-models}{%
\subsubsection{Tsyganenko models}\label{tsyganenko-models}}

The Tsygnanko models are statistical models of the magnetospheric
magnetic field, derived from spacecraft measurements

I've seen several wrappers for this as well. Now I'm trying one called
\emph{geopack} (https://github.com/tsssss/geopack)

    \begin{Verbatim}[commandchars=\\\{\}]
{\color{incolor}In [{\color{incolor}52}]:} \PY{k+kn}{from} \PY{n+nn}{geopack} \PY{k}{import} \PY{n}{geopack}
         \PY{n}{x\PYZus{}gsm}\PY{p}{,}\PY{n}{y\PYZus{}gsm}\PY{p}{,}\PY{n}{z\PYZus{}gsm} \PY{o}{=} \PY{o}{\PYZhy{}}\PY{o}{.}\PY{l+m+mi}{1}\PY{p}{,} \PY{l+m+mi}{0}\PY{p}{,} \PY{o}{\PYZhy{}}\PY{l+m+mf}{1.5}
         
         \PY{c+c1}{\PYZsh{} initialize with universal time (to set dipole tilt I think)}
         \PY{n}{geopack}\PY{o}{.}\PY{n}{recalc}\PY{p}{(}\PY{l+m+mf}{10.}\PY{p}{)}
         
         \PY{n}{x}\PY{p}{,}\PY{n}{y}\PY{p}{,}\PY{n}{z}\PY{p}{,}\PY{n}{xx}\PY{p}{,}\PY{n}{yy}\PY{p}{,}\PY{n}{zz} \PY{o}{=} \PY{n}{geopack}\PY{o}{.}\PY{n}{trace}\PY{p}{(}\PY{n}{x\PYZus{}gsm}\PY{p}{,} \PY{n}{y\PYZus{}gsm}\PY{p}{,} \PY{n}{z\PYZus{}gsm}\PY{p}{,} \PY{n+nb}{dir}\PY{o}{=}\PY{o}{\PYZhy{}}\PY{l+m+mi}{1}\PY{p}{,} \PY{n}{rlim}\PY{o}{=}\PY{l+m+mi}{100}\PY{p}{,} \PY{n}{r0}\PY{o}{=}\PY{o}{.}\PY{l+m+mi}{99999}\PY{p}{,}
                                        \PY{n}{parmod}\PY{o}{=}\PY{l+m+mi}{2}\PY{p}{,}\PY{n}{exname}\PY{o}{=}\PY{l+s+s1}{\PYZsq{}}\PY{l+s+s1}{t89}\PY{l+s+s1}{\PYZsq{}}\PY{p}{,}\PY{n}{inname}\PY{o}{=}\PY{l+s+s1}{\PYZsq{}}\PY{l+s+s1}{igrf}\PY{l+s+s1}{\PYZsq{}}\PY{p}{,}\PY{n}{maxloop}\PY{o}{=}\PY{l+m+mi}{10000}\PY{p}{)}
         \PY{c+c1}{\PYZsh{} (the parmod variable could contain external conditions)}
         \PY{n}{plt}\PY{o}{.}\PY{n}{plot}\PY{p}{(}\PY{o}{\PYZhy{}}\PY{n}{xx}\PY{p}{,} \PY{n}{zz}\PY{p}{)}
         
         \PY{c+c1}{\PYZsh{} plot Eearth}
         \PY{n}{a} \PY{o}{=} \PY{n}{np}\PY{o}{.}\PY{n}{linspace}\PY{p}{(}\PY{l+m+mi}{0}\PY{p}{,} \PY{n}{np}\PY{o}{.}\PY{n}{pi} \PY{o}{*} \PY{l+m+mi}{2}\PY{p}{,} \PY{l+m+mi}{200}\PY{p}{)}
         \PY{n}{plt}\PY{o}{.}\PY{n}{plot}\PY{p}{(}\PY{n}{np}\PY{o}{.}\PY{n}{cos}\PY{p}{(}\PY{n}{a}\PY{p}{)}\PY{p}{,} \PY{n}{np}\PY{o}{.}\PY{n}{sin}\PY{p}{(}\PY{n}{a}\PY{p}{)}\PY{p}{,} \PY{n}{color} \PY{o}{=} \PY{l+s+s1}{\PYZsq{}}\PY{l+s+s1}{black}\PY{l+s+s1}{\PYZsq{}}\PY{p}{)}
         \PY{n}{plt}\PY{o}{.}\PY{n}{show}\PY{p}{(}\PY{p}{)}
\end{Verbatim}


    \begin{Verbatim}[commandchars=\\\{\}]
Load IGRF coefficients {\ldots}

    \end{Verbatim}

    \begin{center}
    \adjustimage{max size={0.9\linewidth}{0.9\paperheight}}{output_80_1.png}
    \end{center}
    { \hspace*{\fill} \\}
    
    \hypertarget{amps}{%
\subsubsection{AMPS}\label{amps}}

The Average Magnetic field and Polar current System model

Python forward code: \emph{pyAMPS} (https://github.com/klaundal/pyAMPS)

    \begin{Verbatim}[commandchars=\\\{\}]
{\color{incolor}In [{\color{incolor}53}]:} \PY{k+kn}{import} \PY{n+nn}{pyamps}
         \PY{n}{m} \PY{o}{=} \PY{n}{pyamps}\PY{o}{.}\PY{n}{AMPS}\PY{p}{(}\PY{l+m+mi}{300}\PY{p}{,} \PY{l+m+mi}{5}\PY{p}{,} \PY{o}{\PYZhy{}}\PY{l+m+mi}{4}\PY{p}{,} \PY{l+m+mi}{23}\PY{p}{,} \PY{l+m+mi}{100}\PY{p}{)} \PY{c+c1}{\PYZsh{} initialize with SW speed, By, Bz, tilt, F107}
         \PY{n}{m}\PY{o}{.}\PY{n}{plot\PYZus{}currents}\PY{p}{(}\PY{p}{)}
\end{Verbatim}


    \begin{center}
    \adjustimage{max size={0.9\linewidth}{0.9\paperheight}}{output_82_0.png}
    \end{center}
    { \hspace*{\fill} \\}
    
    \hypertarget{magnetic-coordinates}{%
\subsubsection{Magnetic coordinates}\label{magnetic-coordinates}}

I will demonstrate apexpy and aacgmv2. Both Apex and AACGM coordinates
involve tracing of magnetic field lines in the IGRF model, which is
computationally heavy and not something you should implement yourself
(although you \emph{could}, using the geopack library and Runge-Kutta
integrators from Scipy). apexpy is a wrapper for Fortran code written by
Emmert et al. (2010), and aacgmv2 is a wrapper for C code written by
Simon Shepherd (2014).

    \begin{Verbatim}[commandchars=\\\{\}]
{\color{incolor}In [{\color{incolor}54}]:} \PY{k+kn}{import} \PY{n+nn}{apexpy}
         
         \PY{n}{now} \PY{o}{=} \PY{n}{datetime}\PY{o}{.}\PY{n}{utcnow}\PY{p}{(}\PY{p}{)}
         \PY{n}{a} \PY{o}{=} \PY{n}{apexpy}\PY{o}{.}\PY{n}{Apex}\PY{p}{(}\PY{n}{date} \PY{o}{=} \PY{n}{now}\PY{p}{,} \PY{n}{refh} \PY{o}{=} \PY{l+m+mi}{110}\PY{p}{)}
         \PY{n}{mlat}\PY{p}{,} \PY{n}{mlon} \PY{o}{=} \PY{n}{a}\PY{o}{.}\PY{n}{geo2apex}\PY{p}{(}\PY{n}{lat}\PY{p}{,} \PY{n}{lon}\PY{p}{,} \PY{l+m+mi}{0}\PY{p}{)}
         \PY{n+nb}{print}\PY{p}{(}\PY{l+s+s1}{\PYZsq{}}\PY{l+s+s1}{The current mlat/MLT in Bergen is }\PY{l+s+si}{\PYZpc{}.1f}\PY{l+s+s1}{ degrees / }\PY{l+s+si}{\PYZpc{}.1f}\PY{l+s+s1}{\PYZsq{}} \PY{o}{\PYZpc{}} 
               \PY{p}{(}\PY{n}{mlat}\PY{p}{,} \PY{n}{a}\PY{o}{.}\PY{n}{mlon2mlt}\PY{p}{(}\PY{n}{mlon}\PY{p}{,} \PY{n}{now}\PY{p}{)}\PY{p}{)}\PY{p}{)}
         
         \PY{c+c1}{\PYZsh{} calcualte magnetic circle of latitude which goes through Bergen}
         \PY{n}{mlons} \PY{o}{=} \PY{n}{np}\PY{o}{.}\PY{n}{linspace}\PY{p}{(}\PY{o}{\PYZhy{}}\PY{l+m+mi}{180}\PY{p}{,} \PY{l+m+mi}{180}\PY{p}{,} \PY{l+m+mi}{360}\PY{p}{)}
         \PY{n}{glat}\PY{p}{,} \PY{n}{glon}\PY{p}{,} \PY{n}{error} \PY{o}{=} \PY{n}{a}\PY{o}{.}\PY{n}{apex2geo}\PY{p}{(}\PY{n}{mlat}\PY{p}{,} \PY{n}{mlons}\PY{p}{,} \PY{l+m+mi}{0}\PY{p}{)}
\end{Verbatim}


    \begin{Verbatim}[commandchars=\\\{\}]
The current mlat/MLT in Bergen is 57.0 degrees / 8.8

    \end{Verbatim}

    \begin{Verbatim}[commandchars=\\\{\}]
{\color{incolor}In [{\color{incolor}55}]:} \PY{k+kn}{import} \PY{n+nn}{aacgmv2}
         
         \PY{n}{mlat\PYZus{}aacgm}\PY{p}{,} \PY{n}{mlon\PYZus{}aacgm} \PY{o}{=} \PY{n}{aacgmv2}\PY{o}{.}\PY{n}{convert}\PY{p}{(}\PY{n}{lat}\PY{p}{,} \PY{n}{lon}\PY{p}{,} \PY{l+m+mi}{0}\PY{p}{,} \PY{n}{now}\PY{p}{,} \PY{n}{a2g} \PY{o}{=} \PY{k+kc}{False}\PY{p}{)}
         \PY{n}{glat\PYZus{}aacgm}\PY{p}{,} \PY{n}{glon\PYZus{}aacgm} \PY{o}{=} \PY{n}{aacgmv2}\PY{o}{.}\PY{n}{convert}\PY{p}{(}\PY{n}{mlat\PYZus{}aacgm}\PY{p}{,} \PY{n}{mlons}\PY{p}{,} \PY{l+m+mi}{0}\PY{p}{,} \PY{n}{now}\PY{p}{,} \PY{n}{a2g} \PY{o}{=} \PY{k+kc}{True}\PY{p}{)}
\end{Verbatim}


    \begin{Verbatim}[commandchars=\\\{\}]
{\color{incolor}In [{\color{incolor}56}]:} \PY{c+c1}{\PYZsh{} plot the latitude circles on a map}
         \PY{n}{fig} \PY{o}{=} \PY{n}{plt}\PY{o}{.}\PY{n}{figure}\PY{p}{(}\PY{n}{figsize} \PY{o}{=} \PY{p}{(}\PY{l+m+mi}{8}\PY{p}{,} \PY{l+m+mi}{7}\PY{p}{)}\PY{p}{)}
         \PY{n}{ax} \PY{o}{=} \PY{n}{fig}\PY{o}{.}\PY{n}{add\PYZus{}subplot}\PY{p}{(}\PY{l+m+mi}{111}\PY{p}{)}
         \PY{n}{m} \PY{o}{=} \PY{n}{Basemap}\PY{p}{(}\PY{n}{projection} \PY{o}{=} \PY{l+s+s1}{\PYZsq{}}\PY{l+s+s1}{ortho}\PY{l+s+s1}{\PYZsq{}}\PY{p}{,} \PY{n}{lon\PYZus{}0} \PY{o}{=} \PY{l+m+mi}{0}\PY{p}{,} \PY{n}{lat\PYZus{}0} \PY{o}{=} \PY{l+m+mi}{90}\PY{p}{,} \PY{n}{ax} \PY{o}{=} \PY{n}{ax}\PY{p}{)}
         \PY{n}{m}\PY{o}{.}\PY{n}{plot}\PY{p}{(}\PY{n}{glon}\PY{p}{,} \PY{n}{glat}\PY{p}{,} \PY{n}{latlon} \PY{o}{=} \PY{k+kc}{True}\PY{p}{,} \PY{n}{linewidth} \PY{o}{=} \PY{l+m+mi}{5}\PY{p}{,} \PY{n}{color} \PY{o}{=} \PY{l+s+s1}{\PYZsq{}}\PY{l+s+s1}{C1}\PY{l+s+s1}{\PYZsq{}}\PY{p}{,} \PY{n}{label} \PY{o}{=} \PY{l+s+s1}{\PYZsq{}}\PY{l+s+s1}{Apex}\PY{l+s+s1}{\PYZsq{}}\PY{p}{)}
         \PY{n}{m}\PY{o}{.}\PY{n}{plot}\PY{p}{(}\PY{n}{glon\PYZus{}aacgm}\PY{p}{,} \PY{n}{glat\PYZus{}aacgm}\PY{p}{,} \PY{n}{latlon} \PY{o}{=} \PY{k+kc}{True}\PY{p}{,} \PY{n}{linewidth} \PY{o}{=} \PY{l+m+mi}{2}\PY{p}{,} \PY{n}{color} \PY{o}{=} \PY{l+s+s1}{\PYZsq{}}\PY{l+s+s1}{C2}\PY{l+s+s1}{\PYZsq{}}\PY{p}{,} \PY{n}{label} \PY{o}{=} \PY{l+s+s1}{\PYZsq{}}\PY{l+s+s1}{AACGM}\PY{l+s+s1}{\PYZsq{}}\PY{p}{)}
         \PY{n}{m}\PY{o}{.}\PY{n}{drawparallels}\PY{p}{(}\PY{p}{[}\PY{l+m+mi}{20}\PY{p}{,} \PY{l+m+mi}{40}\PY{p}{,} \PY{l+m+mi}{60}\PY{p}{,} \PY{l+m+mi}{80}\PY{p}{]}\PY{p}{)}
         \PY{n}{m}\PY{o}{.}\PY{n}{drawmeridians}\PY{p}{(}\PY{n}{np}\PY{o}{.}\PY{n}{r\PYZus{}}\PY{p}{[}\PY{l+m+mi}{0}\PY{p}{:}\PY{l+m+mi}{360}\PY{p}{:}\PY{l+m+mi}{60}\PY{p}{]}\PY{p}{)}
         \PY{n}{m}\PY{o}{.}\PY{n}{drawcoastlines}\PY{p}{(}\PY{n}{color} \PY{o}{=} \PY{l+s+s1}{\PYZsq{}}\PY{l+s+s1}{grey}\PY{l+s+s1}{\PYZsq{}}\PY{p}{)}
         
         \PY{n}{ax}\PY{o}{.}\PY{n}{legend}\PY{p}{(}\PY{n}{frameon} \PY{o}{=} \PY{k+kc}{False}\PY{p}{,} \PY{n}{fontsize} \PY{o}{=} \PY{l+m+mi}{12}\PY{p}{)}
         \PY{n}{plt}\PY{o}{.}\PY{n}{show}\PY{p}{(}\PY{p}{)}
\end{Verbatim}


    \begin{center}
    \adjustimage{max size={0.9\linewidth}{0.9\paperheight}}{output_86_0.png}
    \end{center}
    { \hspace*{\fill} \\}
    
    \hypertarget{other-useful-space-physics-python-projects}{%
\subsubsection{Other useful space physics Python
projects}\label{other-useful-space-physics-python-projects}}

\begin{itemize}
\tightlist
\item
  DaViTpy: \textbf{Extensive} module for working with SuperDARN (does
  not support Python 3) (https://github.com/vtsuperdarn/davitpy)
\item
  pyEphem: For calculating position of astronomical bodies
\item
  SpacePy: Extensive library which includes CDF handling, coordinate
  conversion, and tools for working with Space Weather Modeling
  Framework output
\item
  OvationPyme: Python translation of Ovation Prime (I have not been able
  to run this)
\item
  MadrigalWeb: Access data from Madrigal database
\item
  viresclient: Download Swarm data and evaluate models (including CHAOS,
  IGRF, AMPS (I think) \ldots{})
\end{itemize}

    \hypertarget{pytt-pyrkeland-dipole}{%
\subsubsection{Pytt / Pyrkeland / dipole}\label{pytt-pyrkeland-dipole}}

Some examples from my own code - Pytt (misc):
https://github.com/klaundal/pytt (private repository, but I will give
you access if you want) - dipole: for (extremely fast) calculation of
dipole tilt angle

    \begin{Verbatim}[commandchars=\\\{\}]
{\color{incolor}In [{\color{incolor}57}]:} \PY{c+c1}{\PYZsh{} plotting on MLT mlat grid:}
         \PY{k+kn}{from} \PY{n+nn}{pytt}\PY{n+nn}{.}\PY{n+nn}{plotting} \PY{k}{import} \PY{n}{polarsubplot}
         
         \PY{n}{ax} \PY{o}{=} \PY{n}{plt}\PY{o}{.}\PY{n}{figure}\PY{p}{(}\PY{p}{)}\PY{o}{.}\PY{n}{add\PYZus{}subplot}\PY{p}{(}\PY{l+m+mi}{111}\PY{p}{)}
         \PY{n}{pax} \PY{o}{=} \PY{n}{polarsubplot}\PY{o}{.}\PY{n}{Polarsubplot}\PY{p}{(}\PY{n}{ax}\PY{p}{,} \PY{n}{linestyle} \PY{o}{=} \PY{l+s+s1}{\PYZsq{}}\PY{l+s+s1}{:}\PY{l+s+s1}{\PYZsq{}}\PY{p}{,} \PY{n}{color} \PY{o}{=} \PY{l+s+s1}{\PYZsq{}}\PY{l+s+s1}{grey}\PY{l+s+s1}{\PYZsq{}}\PY{p}{)}
         \PY{n}{mlat}\PY{p}{,} \PY{n}{mlt} \PY{o}{=} \PY{n}{np}\PY{o}{.}\PY{n}{ones}\PY{p}{(}\PY{l+m+mi}{100}\PY{p}{)} \PY{o}{*} \PY{l+m+mi}{57}\PY{p}{,} \PY{n}{np}\PY{o}{.}\PY{n}{linspace}\PY{p}{(}\PY{l+m+mi}{18}\PY{p}{,} \PY{l+m+mi}{24} \PY{o}{+} \PY{l+m+mi}{6}\PY{p}{,} \PY{l+m+mi}{100}\PY{p}{)}
         \PY{n}{pax}\PY{o}{.}\PY{n}{plot}\PY{p}{(}\PY{n}{mlat}\PY{p}{,} \PY{n}{mlt}\PY{p}{,} \PY{n}{linewidth} \PY{o}{=} \PY{l+m+mi}{4}\PY{p}{)}
         \PY{n}{pax}\PY{o}{.}\PY{n}{scatter}\PY{p}{(}\PY{p}{[}\PY{l+m+mi}{70}\PY{p}{,} \PY{l+m+mi}{70}\PY{p}{]}\PY{p}{,} \PY{p}{[}\PY{l+m+mi}{15}\PY{p}{,} \PY{l+m+mi}{9}\PY{p}{]}\PY{p}{,} \PY{n}{marker} \PY{o}{=} \PY{l+s+s1}{\PYZsq{}}\PY{l+s+s1}{o}\PY{l+s+s1}{\PYZsq{}}\PY{p}{,} \PY{n}{s} \PY{o}{=} \PY{l+m+mi}{100}\PY{p}{)}
         \PY{n}{plt}\PY{o}{.}\PY{n}{show}\PY{p}{(}\PY{p}{)}
\end{Verbatim}


    \begin{center}
    \adjustimage{max size={0.9\linewidth}{0.9\paperheight}}{output_89_0.png}
    \end{center}
    { \hspace*{\fill} \\}
    
    \begin{Verbatim}[commandchars=\\\{\}]
{\color{incolor}In [{\color{incolor}58}]:} \PY{c+c1}{\PYZsh{} dipole tilt angle }
         \PY{k+kn}{from} \PY{n+nn}{dipole} \PY{k}{import} \PY{n}{dipole\PYZus{}tilt}
         
         \PY{c+c1}{\PYZsh{} calculate tilt angle for each time in the SuperMAG }
         \PY{c+c1}{\PYZsh{} file that was loaded above (called data)}
         \PY{n}{tilt} \PY{o}{=} \PY{n}{dipole\PYZus{}tilt}\PY{p}{(}\PY{n}{data}\PY{o}{.}\PY{n}{index}\PY{p}{)}
         \PY{n}{plt}\PY{o}{.}\PY{n}{plot}\PY{p}{(}\PY{n}{data}\PY{o}{.}\PY{n}{index}\PY{p}{,} \PY{n}{tilt}\PY{p}{)}
         \PY{n}{plt}\PY{o}{.}\PY{n}{show}\PY{p}{(}\PY{p}{)}
\end{Verbatim}


    \begin{center}
    \adjustimage{max size={0.9\linewidth}{0.9\paperheight}}{output_90_0.png}
    \end{center}
    { \hspace*{\fill} \\}
    
    \begin{Verbatim}[commandchars=\\\{\}]
{\color{incolor}In [{\color{incolor}59}]:} \PY{c+c1}{\PYZsh{} conversion between geodetic and geocentric coordinates}
         \PY{c+c1}{\PYZsh{} geodetic takes into account ellipsoid \PYZhy{} this is what is normally used in maps}
         \PY{k+kn}{from} \PY{n+nn}{pytt}\PY{n+nn}{.}\PY{n+nn}{earth}\PY{n+nn}{.}\PY{n+nn}{geodesy} \PY{k}{import} \PY{n}{geod2geoc}
         
         \PY{n}{theta}\PY{p}{,} \PY{n}{r}\PY{p}{,} \PY{n}{south}\PY{p}{,} \PY{n}{radial} \PY{o}{=} \PY{n}{geod2geoc}\PY{p}{(}\PY{n}{lat}\PY{p}{,} \PY{l+m+mi}{0}\PY{p}{,} \PY{l+m+mi}{0}\PY{p}{,} \PY{o}{\PYZhy{}}\PY{l+m+mi}{1}\PY{p}{)} 
         \PY{n+nb}{print}\PY{p}{(}\PY{l+s+s1}{\PYZsq{}}\PY{l+s+s1}{geocentric latitude: }\PY{l+s+si}{\PYZpc{}.1f}\PY{l+s+s1}{, geodetic latitude: }\PY{l+s+si}{\PYZpc{}.1f}\PY{l+s+s1}{\PYZsq{}} \PY{o}{\PYZpc{}} \PY{p}{(}\PY{l+m+mi}{90} \PY{o}{\PYZhy{}} \PY{n}{theta}\PY{p}{,} \PY{n}{lat}\PY{p}{)}\PY{p}{)}
         \PY{n+nb}{print}\PY{p}{(}\PY{l+s+s1}{\PYZsq{}}\PY{l+s+s1}{geocentric radius: }\PY{l+s+si}{\PYZpc{}.1f}\PY{l+s+s1}{ km, compared to mean Earth radius }\PY{l+s+si}{\PYZpc{}.1f}\PY{l+s+s1}{ km}\PY{l+s+s1}{\PYZsq{}} \PY{o}{\PYZpc{}} \PY{p}{(}\PY{n}{r}\PY{p}{,} \PY{l+m+mf}{6371.2}\PY{p}{)}\PY{p}{)}
         \PY{n+nb}{print}\PY{p}{(}\PY{l+s+s1}{\PYZsq{}}\PY{l+s+s1}{An upward unit vector has radial component }\PY{l+s+si}{\PYZpc{}.7f}\PY{l+s+s1}{ and southward component }\PY{l+s+si}{\PYZpc{}.7f}\PY{l+s+s1}{ in a geocentric system}\PY{l+s+s1}{\PYZsq{}} \PY{o}{\PYZpc{}} \PY{p}{(}\PY{n}{radial}\PY{p}{,} \PY{n}{south}\PY{p}{)}\PY{p}{)}
\end{Verbatim}


    \begin{Verbatim}[commandchars=\\\{\}]
geocentric latitude: 60.2, geodetic latitude: 60.4
geocentric radius: 6362.0 km, compared to mean Earth radius 6371.2 km
An upward unit vector has radial component 0.9999958 and southward component -0.0028904 in a geocentric system

    \end{Verbatim}

    \begin{Verbatim}[commandchars=\\\{\}]
{\color{incolor}In [{\color{incolor}60}]:} \PY{c+c1}{\PYZsh{} use biot savart to evaluate field of line current on sphere}
         \PY{k+kn}{from} \PY{n+nn}{pytt}\PY{n+nn}{.}\PY{n+nn}{mag}\PY{n+nn}{.}\PY{n+nn}{biot\PYZus{}savart} \PY{k}{import} \PY{n}{biot\PYZus{}savart}
         \PY{n}{RE} \PY{o}{=} \PY{l+m+mf}{6371.2} \PY{o}{*} \PY{l+m+mf}{1e3}
         \PY{n}{current\PYZus{}height} \PY{o}{=} \PY{n}{RE} \PY{o}{+} \PY{l+m+mi}{100} \PY{o}{*} \PY{l+m+mf}{1e3}
         \PY{n}{amplitude} \PY{o}{=} \PY{l+m+mi}{100} \PY{o}{*} \PY{l+m+mf}{1e3} \PY{c+c1}{\PYZsh{} Amperes}
         
         \PY{c+c1}{\PYZsh{} make a grid}
         \PY{n}{mlatxx}\PY{p}{,} \PY{n}{mltxx} \PY{o}{=} \PY{n}{np}\PY{o}{.}\PY{n}{meshgrid}\PY{p}{(}\PY{n}{np}\PY{o}{.}\PY{n}{linspace}\PY{p}{(}\PY{l+m+mi}{50}\PY{p}{,} \PY{l+m+mi}{90}\PY{p}{,} \PY{l+m+mi}{70}\PY{p}{)}\PY{p}{,} \PY{n}{np}\PY{o}{.}\PY{n}{linspace}\PY{p}{(}\PY{l+m+mi}{0}\PY{p}{,} \PY{l+m+mi}{24}\PY{p}{,} \PY{l+m+mi}{100}\PY{p}{)}\PY{p}{)}
         
         \PY{n}{B} \PY{o}{=} \PY{n}{biot\PYZus{}savart}\PY{p}{(}\PY{n}{mlat}\PY{p}{,} \PY{n}{mlt}\PY{p}{,} \PY{n}{mlatxx}\PY{p}{,} \PY{n}{mltxx}\PY{p}{,} \PY{n}{amplitude}\PY{p}{,} \PY{n}{RB} \PY{o}{=} \PY{n}{RE}\PY{p}{,} \PY{n}{RI} \PY{o}{=} \PY{n}{current\PYZus{}height}\PY{p}{)}
         \PY{n}{B} \PY{o}{=} \PY{n}{np}\PY{o}{.}\PY{n}{linalg}\PY{o}{.}\PY{n}{norm}\PY{p}{(}\PY{n}{B}\PY{p}{,} \PY{n}{axis} \PY{o}{=} \PY{l+m+mi}{0}\PY{p}{)}
         
         \PY{n}{ax} \PY{o}{=} \PY{n}{plt}\PY{o}{.}\PY{n}{figure}\PY{p}{(}\PY{p}{)}\PY{o}{.}\PY{n}{add\PYZus{}subplot}\PY{p}{(}\PY{l+m+mi}{111}\PY{p}{)}
         \PY{n}{pax} \PY{o}{=} \PY{n}{polarsubplot}\PY{o}{.}\PY{n}{Polarsubplot}\PY{p}{(}\PY{n}{ax}\PY{p}{,} \PY{n}{linestyle} \PY{o}{=} \PY{l+s+s1}{\PYZsq{}}\PY{l+s+s1}{:}\PY{l+s+s1}{\PYZsq{}}\PY{p}{,} \PY{n}{color} \PY{o}{=} \PY{l+s+s1}{\PYZsq{}}\PY{l+s+s1}{grey}\PY{l+s+s1}{\PYZsq{}}\PY{p}{)}
         \PY{n}{c} \PY{o}{=} \PY{n}{pax}\PY{o}{.}\PY{n}{contourf}\PY{p}{(}\PY{n}{mlatxx}\PY{p}{,} \PY{n}{mltxx}\PY{p}{,} \PY{n}{B} \PY{o}{*} \PY{l+m+mf}{1e9}\PY{p}{)}
         \PY{n}{plt}\PY{o}{.}\PY{n}{colorbar}\PY{p}{(}\PY{n}{c}\PY{p}{,} \PY{n}{label} \PY{o}{=} \PY{l+s+s1}{\PYZsq{}}\PY{l+s+s1}{nT}\PY{l+s+s1}{\PYZsq{}}\PY{p}{)}
         \PY{n}{pax}\PY{o}{.}\PY{n}{scatter}\PY{p}{(}\PY{n}{mlat}\PY{p}{[}\PY{l+m+mi}{0}\PY{p}{]} \PY{p}{,} \PY{n}{mlt}\PY{p}{[}\PY{l+m+mi}{0}\PY{p}{]}\PY{p}{,} \PY{n}{marker} \PY{o}{=} \PY{l+s+s1}{\PYZsq{}}\PY{l+s+s1}{o}\PY{l+s+s1}{\PYZsq{}}\PY{p}{,} \PY{n}{s} \PY{o}{=} \PY{l+m+mi}{100}\PY{p}{,} \PY{n}{c} \PY{o}{=} \PY{l+s+s1}{\PYZsq{}}\PY{l+s+s1}{blue}\PY{l+s+s1}{\PYZsq{}}\PY{p}{)}
         \PY{n}{pax}\PY{o}{.}\PY{n}{scatter}\PY{p}{(}\PY{n}{mlat}\PY{p}{[}\PY{o}{\PYZhy{}}\PY{l+m+mi}{1}\PY{p}{]}\PY{p}{,} \PY{n}{mlt}\PY{p}{[}\PY{o}{\PYZhy{}}\PY{l+m+mi}{1}\PY{p}{]}\PY{p}{,} \PY{n}{marker} \PY{o}{=} \PY{l+s+s1}{\PYZsq{}}\PY{l+s+s1}{o}\PY{l+s+s1}{\PYZsq{}}\PY{p}{,} \PY{n}{s} \PY{o}{=} \PY{l+m+mi}{100}\PY{p}{,} \PY{n}{c} \PY{o}{=} \PY{l+s+s1}{\PYZsq{}}\PY{l+s+s1}{red}\PY{l+s+s1}{\PYZsq{}}\PY{p}{)}
         \PY{n}{plt}\PY{o}{.}\PY{n}{show}\PY{p}{(}\PY{p}{)}
\end{Verbatim}


    \begin{center}
    \adjustimage{max size={0.9\linewidth}{0.9\paperheight}}{output_92_0.png}
    \end{center}
    { \hspace*{\fill} \\}
    
    \begin{Verbatim}[commandchars=\\\{\}]
{\color{incolor}In [{\color{incolor}61}]:} \PY{k+kn}{import} \PY{n+nn}{antigravity}
\end{Verbatim}


    

    \begin{Verbatim}[commandchars=\\\{\}]
{\color{incolor}In [{\color{incolor}62}]:} \PY{n}{plt}\PY{o}{.}\PY{n}{rcParams}\PY{p}{[}\PY{l+s+s1}{\PYZsq{}}\PY{l+s+s1}{text.usetex}\PY{l+s+s1}{\PYZsq{}}\PY{p}{]} \PY{o}{=} \PY{k+kc}{False}
         \PY{n}{plt}\PY{o}{.}\PY{n}{xkcd}\PY{p}{(}\PY{p}{)}
         \PY{n}{plt}\PY{o}{.}\PY{n}{plot}\PY{p}{(}\PY{p}{[}\PY{l+m+mi}{0}\PY{p}{,} \PY{l+m+mi}{1}\PY{p}{]}\PY{p}{,} \PY{p}{[}\PY{l+m+mi}{0}\PY{p}{,} \PY{l+m+mi}{1}\PY{p}{]}\PY{p}{)}
         \PY{n}{plt}\PY{o}{.}\PY{n}{show}\PY{p}{(}\PY{p}{)}
\end{Verbatim}


    \begin{center}
    \adjustimage{max size={0.9\linewidth}{0.9\paperheight}}{output_95_0.png}
    \end{center}
    { \hspace*{\fill} \\}
    
    

    \hypertarget{the-end}{%
\section{The end}\label{the-end}}

Send me an email to get access to pytt and this notebook on github


    % Add a bibliography block to the postdoc
    
    
    
    \end{document}
